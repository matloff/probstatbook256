\documentclass[11pt]{article}  

\setlength{\oddsidemargin}{0in}
\setlength{\evensidemargin}{0in}
\setlength{\topmargin}{0.0in}
\setlength{\headheight}{0in}
\setlength{\headsep}{0in}
\setlength{\textwidth}{6.5in}
\setlength{\textheight}{9.0in}
\setlength{\parindent}{0in}
\setlength{\parskip}{0.1in}

\usepackage{times}
\usepackage{hyperref}
\usepackage{fancyvrb}
\usepackage{relsize}  
\usepackage{graphicx}

\begin{document}

\title{Professor Norm Matloff's Beginner's Guide to Installing and
Using Linux
\thanks{The information contained here is accurate, to the author's
knowledge.  However, no guarantee is made in this regard.}
}

\author{
Dr. Norm Matloff \\
Department of Computer Science \\
University of California at Davis \\
matloff@cs.ucdavis.edu \\
\copyright{1999-2008} }
\date{September 19, 2008} 

\maketitle

\tableofcontents{}

\newpage

\section{Overview}

\subsection{Background Needed}

I have tailored the material here to beginners. No special
sophistication in computers is needed. Any typical Microsoft Windows
user should be able to understand the instructions here and install
Linux in less than an hour's time. (Do not be intimidated by the length
of this document; you probably will not have to use most of it.)

\subsection{What Is Linux?}

Linux is a form of the Unix operating system.  Though originally Unix
was used mainly by engineers and scientists and thus was not very
familiar to the general public, a lot of what you take for granted on
computer systems today began in Unix.  A notable example is the
Internet---the first major operating system to implement the
TCP/IP protocol at the heart of the Internet was Unix, and that led to
the general acceptance of the protocol.

In the early 1990s, computer science student Linus Torvalds decided to
write his own version of Unix, which he called Linux.  Other
``homegrown'' versions of Unix had been written, such as MINIX, but what
distinguished Linux was the scale of worldwide participation involved.
Torvalds innocently put a message on the Internet asking if anyone
wanted to help, and he got a torrent of responses.

There are a several reasons why Linux is mainstream today.  First, it
became known as a very reliable, stable operating system, with one
result being that Linux has become a major platform for large corporate
Web servers.  Another reason is that it, and the vast majority of the
software associated with it developed elsewhere, is free.  Many
companies have found that it is cheaper to run Linux on their PCs, both
for this reason and because of reduced maintenance costs.

There are several good reasons for you to use Linux:

\begin{itemize}

\item As mentioned, Linux is becoming one of the ``hottest'' software
systems. Virtually all of the major companies---IBM, HP, Sun
Microsystems, etc.---are promoting it, and as mentioned Linux is a
leading corporate choice for Web servers.  Linux is the main operating
system used at Google.

\item Linux is also starting to make inroads in large desktop markets,
such as businesses, schools and so on, due to its high reliability, far
lower rate of infection by viruses compared to Windows, and its low
cost.

\item The Linux community shares.  That means that people online are
much more willing to help you (see Section \ref{help}), and more open
source software is available.

\end{itemize}

If you are a university computer science student, there are some very
important additional advantages: 

\begin{itemize}

\item Many CS courses make specific use of Unix, and thus their work
cannot be done on Windows platforms. Since it is a full Unix system,
Linux allows students to do their homework in the comfort of their own
homes. If you are new to Unix, click here for my Unix tutorial Web page
at \url{http://heather.cs.ucdavis.edu/~matloff/unix.html}, which will
introduce you to Unix file and directory commands, and so on.

\item In installing and using Linux, students learn many practical
things about computers which they do not learn in coursework. This
practical experience can also help you in job interviews, both for
permanent jobs after graduation and for summer jobs and
internships/co-ops during your college years. Even if the job you
interview for does not involve Linux, you will definitely impress the
interviewer if, for example, you discuss various things you have done to
use and customize your Linux system.

\end{itemize}

\section{Linux Distributions}

Linux comes in various {\it distributions}, called {\it distros} by
Linux afficionados---but they are all Linux in terms of functionality.
Some of the most popular are Ubuntu, Red Hat, Fedora, SuSE, MEPIS,
PCLinuxOS and so on.

\subsection{Which One Is Best?}

Remember, there are tons of good distros out there.  Any of the above
would be fine, as would many others, but here is the short answer:  {\bf
\large Use Ubuntu.}  It is arguably one of the most user-friendly of the
distros, and it has a large user community you can access in the Ubuntu
forum on the Web, probably the most active one out there.  

I now use Ubuntu myself on my home computers, as well as on my office
computer, after years of using the Fedora/Red Hat Linux distros.  I find
that Ubuntu's package installation works much better, for example.

If you have an old machine, especially one with limited memory (i.e.
RAM), you may wish to give Puppy Linux or Damn Small Linux a try.  I
installed them (one at a time) on an old 1998 laptop with only 64M of
memory!  And they take as little as 50M of disk space.  See Section
\ref{ancient} for details.

\subsection{Obtaining Linux}

You can obtain your desired distro (assuming it's one of the free
distros, such as Ubuntu)
by downloading from the Web and burning a CD (its
basic installation form is small enough to fit on a CD).

Or you can buy a book devoted to the distro, or buy a  Linux magazine
that includes a CD for it.  

Important note:  If you download Linux from the Web and burn it to a CD
or DVD, make sure that you burn the ISO image, as opposed to copying
the ISO file as you would in, say, a backup operation.  Your burner
software should have a choice in its menu for this.  

\subsection{Live-CD Linux Distributions}
\label{live}

A more recent concept in Linux distributions is that of {\it live CD}
distribution.  Here the Linux package is on a bootable CD-ROM.  The user
inserts the CD in the drive, reboots, and then Linux boots up.

The advantage of this approach is that one does not have to get involved
in disk partitioning, a sometimes difficult process.  One is using Linux
without actually installing it, thus without changing the disk
partitioning.

A disadvantage is that it may not allow one's application programs to
save files to the hard drive, unless one has already split the Windows
partition, defeating much of the purpose.  However, one can save files
to a USB key. 

So, the approach ideal for those who wish to just try Linux for a short
period of time, but not so useful for long-term use. 

The first well-known live-CD distribution was Knoppix, but there are
many others today, including Ubuntu, whose CD you can use either as a
live-CD or for permanent installation.  

If you use the live-CD approach, you may of course skip Section
\ref{install} of this tutorial.

\subsection{Live CDs As Rescue Tools}

Among other things, Knoppix has developed a reputation as being useful
as an OS rescue/repair tool, including for Windows!  And now, most of
the live CDs can be used this way.  For details, see a Knoppix book or
search on the Web.

\section{Installing Linux}
\label{install}

\subsection{Assumptions}

\subsubsection{Generality/Specificity of Coverage}

This part of the tutorial will not go into the details for installing
one particular distribution.  That would be impractical, since the
details for any one distribution often change substantially from one
release to the next.  So instead, this section on installation will
discuss the major points you should watch for during the procedure.  It
will sometimes use Ubuntu as an example for concreteness, but the
principles should be similar for most other distributions.

\subsubsection{Your Machine}

It is assumed that you have an Intel-compatible desktop or notebook,
with a bootable CD-ROM or DVD drive.
% \footnote{The latter condition
% should hold for almost any machine bought in the last five or six years.
% One other point, though, is that the boot priority should be set so that
% the machine tries to boot from CD-ROM or DVD before trying to boot from
% the hard drive.  Your machine probably already does this, but if not,
% you can reset the BIOS to do so.  Consult your manual on this, or ask at
% any computer store.}
You should have at least 128M of RAM.  I recommend that you have at
least 10G of disk space available for Linux, though 5G would probably be
enough.  If you have a smaller machine, try one of the distros designed
specifically for that purpose, discussed in Section \ref{ancient}.

\subsection{Determine Your Hardware Details}

The Linux installation program will be able to sense most of your
hardware information.  So, you can probably skip our section here.  But
if you want to take about five minutes extra time here, it could be
helpful later if you write down some of your hardware types before
beginning installation.

You could download the free program {\bf Hardinfo}, and run it to record
a list of your hardware.  Or to check your hardware from Windows XP,
select My Computer $|$ Control Panel $|$ System $|$ Hardware $|$ Device
Manager.  Click General to get the amount of RAM and CPU type. Then go
to Device Manager, and click on the `+' next to each component, e.g.
"Disk drives,'' "Display adapters'' and so on.  Write down the
information, including your hard drive type, such as IDE; your video
card make and model; your monitor make and model; the type of connection
used for your mouse, such as PS/2; the make and model of your printer;
etc. 

Do you still have the manual which came with your monitor?  If so, check
the specs in the back, and write down the horizontal sync and vertical
refresh rate, and the make and model.

\subsection{Partioning Your Hard Drive}
\label{part}

Today most Linux distros, such as Mandriva, SuSE and Ubuntu, do the disk
partitioning for you.  This is a major advantage, as partitioning is a
vital but delicate operation.  Later in this section, I'll give you some
advice for the Ubuntu case, and also give you some options to use if you
have a distro that does not do automatic partitioning.  

But I do suggest that even if you will have automatic partitioning done,
it would still be worthwhile for  you to read Section \ref{whatispart}.
This would be useful both for the installation process and later on in
your role as an ``informed consumer.''

\subsubsection{What Is Partitioning?}
\label{whatispart} 

Again, it is probably not necessary for you to know the material here,
and it is rather detailed, but you may find it useful at some point.  I
do recommend that you take a few minutes and read this section.

A hard drive will consist of one or more {\it partitions}.  A partition
is a set of contiguous space (sequential blocks) on the disk, and is
treated as an independent disk.  

So, assuming you want your system to include both Windows and Linux
(termed a {\it dual boot} situation, since you can boot either system),
you will need at least one partition for Windows and one (actually two)
for Linux.

It's important to understand how the naming works: In Linux systems, all
I/O devices are treated as ``files.''  If your first hard drive is of
the IDE type, the entire drive is probably called {\bf /dev/hda}, i.e.
the ``file'' {\bf hda} within the directory {\bf /dev}.  In the case of
SATA-type hard drives, the notation is {\bf /dev/sda} etc.

Your first CD-ROM/DVD drive is likely {\bf /dev/hdc} (your third ``hard
drive''), your first USB port is likely {\bf /dev/sdf1} and so on.

Partitions within your first drive are called {\bf /dev/hda1}, {\bf
/dev/hda2} and so on.  Your original Windows single partition was
probably {\bf /dev/hda1}.  

Within a partition you'll have some type of file system.  The disk
consists simply of a long stream of bytes, with no structure, so the OS
needs to have a way of organizing them into files, recording where in
that stream each file has its bytes.  But you don't need to know the
details.  Windows XP and Vista use the {\bf NTFS} file system.  The
standard Linux file system is {\bf ext2} (number 0x83, sometimes called
{\it Linux native}), or possibly {\bf ext3}, for your main Linux
partition and of type {\bf swap} for your swap partition (number 0x82,
used for temporary storage during the time the OS is running).

PCs were originally designed to have up to four ``real'' partitions,
called {\it primary} partitions.  After people found that to be too
constraining, {\it logical} or {\it extended} partitions were invented.
You should install Linux in a primary partition, for recovery reasons,
but it is not necessary.

\subsubsection{Before You Do the Partitioning}

Before you start, give some thought as to how much of the original
partition you want to keep for Windows and how much you want to leave
for Linux.  If you plan to become a serious Linux user,\footnote{And as
mentioned in Section \ref{useit}, if you want to learn Linux, the only
way to do it is to become a serious, every day user.} you'll want to
allocate at least half of the space for Linux. 

You really ought to run Windows' {\bf chkdsk} command first, in case you
have any bad sectors on your hard drive.  You may also wish to
defragment.

% \subsubsection{``Exploding cold door''!---Windows Vista Can Do the
% Partitioning for You}
% 
% The Cantonese phrase, {\it bau laang muhn}, literally ``exploding cold
% door,'' means the speaker is surprised to see the least-expected event
% occur.  Given that Linux cuts a bit into Microsoft's business and there
% has been long-held suspicion that Microsoft has tried to put roadblocks
% against Linux and other OSs in its system, it is a pleasant surprise
% that the Windows Vista OS can actually do the repartitioning for you.
% 
% So, instead of some of the methods given below, if you have Vista on
% your system, you might take this route.
% See \url{http://nighthacker.net/how-to-dual-boot-vista-and-ubuntu-gutsy710/}
% for details.
% 
% That will only free up the space.  You then must create your Linux
% partitions within your Linux distro, as discussed below.

\subsubsection{Partitioning Using GParted}

Today most distros will invoke a partitioning program to do your
partitioning.  This could be the famous GParted program, or one that the
authors of your distro wrote themselves.

You can use GParted on your own by downloading and booting a GParted
live CD before you install Linux, but I'll assume here that your Linux
installation program invokes either GParted or another program written
specifically for your distro.

Since every distro will handle this a bit differently, what I will do
here is just give you an understanding of what operations need to be
done, with the specific mouse clicks needed varying from one distro to
another.

I'll assume that you want your Windows and Linux systems to coexist on
the same hard drive.  So when your distro's installer program asks you
whether you want to use the entire disk, be sure to say no!  Of course,
if you do want to erase Windows, or if you are installing Linux on a
separate drive from Windows, you can go ahead and use the whole drive.

% When I installed Ubuntu in place of my existing Fedora system, I chose
% manual partitioning.  Ubuntu said my swap space wasn't large enough, so
% I told Ubuntu to remove my Fedora partitions, and then to create new
% root and swap partitions from the freed space.  I chose the {\bf ext3} 
% filesystem type for the root partition and specified that it should be
% mounted at {\bf /}.

Here are the main steps in GParted, roughly stated (you may see some
variation):

\begin{itemize}

\item Select the disk you wish to repartition.  If you have only one
disk, it will be something like {\bf /dev/hda}.  (See Section
\ref{whatispart}.)

\item Select the partition where Windows resides.  This will typically
be the largest one, and almost certainly of file system type
NTFS.

\item Note how much space is remaining, and decide how much of it you
want to remove from the Windows partition in order to make a partition
for Linux.

\item Now resize, in this case shrink, the Windows partition.  The
partioner will ask you how much room to make.

\item Adjust the partition size according to your desired value.

\item You'll need to make the main Linux partition {\bf primary}, of
type {\bf ext2} or {\bf ext3}, and set to be bootable.

\item You'll need a smaller partition of type {\bf linux-swap}.  This is
not used for files, but rather as ``scratch space'' by the
OS.\footnote{Most modern operating systems, including Windows and Linux,
use {\it virtual memory}.  This allows situations in which the total
memory, i.e. RAM, needed for all of the programs we are currently
running, exceeds the amount of physical memory we have.  This is
accomplished by temporarily placing some of the memory contents on disk,
in the swap partition.  Virtual memory also enables certain safety
features, such as preventing one program from writing to the data of
another.} 

\item You'll then have to commit, i.e. save, the changes to the
partitions.  This might take a few minutes, so be patient.

\item The next time you boot Windows, you will be asked if you want a
disk consistency check.  Definitely say yes.

\end{itemize}

\subsection{The Installation Process} 

\subsubsection{To Begin}

By the way, if you are upgrading or replacing another version or
distribution of Linux, see Section \ref{upgrade} before beginning.

Put your Linux CD-ROM or DVD in the drive, and reboot.  The installation
program should begin.\footnote{\label{bios} If not, you must change the
BIOS settings to make the CD-ROM bootable (and the first device checked
during the boot process); see your computer's manual on how to do this.}

\subsubsection{Questions You \underline{May} Be Asked During the
Installation Process}

The trend in time is for the installation programs to actually ask you
fewer and fewer questions, i.e. the process has become more and more
automated.  Most of the questions discussed in this section will NOT be
asked---Ubuntu will probably ask none of them---but the information here
will give you an idea of how to answer if they are asked.

\begin{itemize}

\item Some distributions will give you a choice of several installation
types, which vary in terms of what kinds of application software will be
installed.  If you are a CS student, you need to make sure your
installation will include compilers, editors, debuggers and so on.  Note
that you can always add more applications later on.  But since most
people now have plenty of disk space, it is easier to simply ask for
everything.  

\item Assuming you'll want a {\it dual-boot system}, i.e. you'll be
having both Windows and Linux available for booting, you need some sort
of {\it boot loader}.  This is a program which upon powerup of your
computer will ask you which OS you wish to boot at that time.  Your
distribution will probably use the GRUB boot loader, or possibly LILO.
It doesn't matter that much for a beginner, but if asked, definitely
indicate that you want to be able to boot both OSs.  (If you are not
asked, the distro should make it dual-boot by default.)  Take the
defaults for everything else, e.g. the choice of bootloader program. 

\item If you're asked whether you want 3-button mouse emulation, say
yes.  If you have only a 2-button mouse (the wheel does count as a
button), this emulation will enable cut-and-paste window operations.

\item You'll need a GUI (``graphical user intrface'') desktop manager.
The two most widely-used GUI desktop managers for Linux are KDE and
GNOME.  Each has its band of devoted followers.  I generally use GNOME
these days, but both are good.  It really doesn't matter which one you
choose for new users, and you can always switch later if desired.
Choose one (or both).

\item I mentioned earlier that disk partitioning has over the years been
one of the two major issues in Linux installation.  The other has been
configuring for the video card and monitor.

With today's modern Linux installation programs, this is typically not a
problem.  They are pretty good at identifying your video card, and
guessing good settings to use.  Typically they will give you a chance to
test those settings out before continuing with the installation process,
with a test image.  My experience has generally been that that is
sufficient.

If that image does not turn out well, the installation program will
typically give you a chance to state the make and model of your video
card, and horizontal sync, vertical refresh rate, and make and model of
your monitor.  That is why I asked earlier if you still have the manual
for your monitor.  (On a laptop, though, you often don't have this
information, since its monitor is built in.)

By the way, once a configuration has been decided on, it will be saved
to a file, such as {\bf /etc/X11/xorg.conf}.  You can look at this later
if you are curious as to what configuration the installer has chosen for
you, and can modify it if you know what needs to be tweaked. 

\item You may be asked if your machine has a static Internet address.
In most cases, the answer should be no; for a home machine or wireless
use you probably get a dynamic Internet address, using a protocol named
DHCP.

\end{itemize}

\subsection{Installing Tiny Linuxes}
\label{ancient}

\section{Post-Installation Configuration}

This section describes some further steps I recommend taking after your
installation is finished. 

\subsection{Help in Hardware Configuration}

Having trouble getting some hardware component to work under Linux?
I'll have some tips on that below, but keep in mind that a great source
is the Web.  Plug something like ``Linux install XXXX,'' where XXXX is
the type of machine you own) into Google.  Actually, it would be better
to specify your distro, e.g. ``Ubuntu install XXXX.'' You'll find a
number of reports of experiences by other people with your
machine/distro.  

\subsection{Configuring Your Search Path (``Why can't I run my a.out?'')}

Most Linux distros do not include your current directory, `.',
in the PATH variable. Thus if for example you compile a program and then
type

\begin{verbatim}
a.out
\end{verbatim}

the shell may tell you that {\bf a.out} is not found. You are expected
to explicitly specify the current directory:

\begin{verbatim}
./a.out
\end{verbatim}

If you consider this a problem, as I do, to remedy it in the case of the
BASH shell (the default shell for most distros), edit the file {\bf
~/.bash\_profile} In the line which sets PATH, append ``:.'' (a colon
and a dot) at the end of the line, with no intervening spaces.  Then log
out and log in again, or do

\begin{verbatim}
source ~/.bash_profile
\end{verbatim}

\subsection{Configuring a Printer}

Your Linux distribution should have some program to help you configure
your printer if something went wrong during installation.  For example,
if you are running the GNOME GUI, select System $|$ Administration $|$
Printing.

\subsection{Wireless Networking}
\label{wifi}

\subsubsection{Connecting}

If you are running the GNOME windows manager, select System $|$
Administration $|$ Network.  In KDE, it's System $|$ Network Device
Control.  

Highlight the entry for your wireless device.  Your WiFi device is
probably {\bf eth1}.  Make sure the box is checked.  Then you'll
probably have to click on Properties or something like that.

The names of wireless access points are called {\bf ESSID}s.  You can
determine which ESSIDs are within range of you by typing the command 

\begin{Verbatim}[fontsize=\relsize{-2}]
$ iwlist scanning
\end{Verbatim}

into a terminal window.  State the ESSID you want.  (Note that some of
the ones listed might be private.)

If you are connected to a router or a wireless access point, you
probably get your IP address via DHCP, rather than statically.

The network managers included with most Linux distros are rather
primitive.  An excellent alternative is WiFi Radar.  In Ubuntu, install
via

\begin{Verbatim}[fontsize=\relsize{-2}]
sudo apt-get install wifi-radar
\end{Verbatim}

\subsubsection{Other Tools}

Your Linux system will provide various tools to configure and monitor
your network:

Useful commands from a terminal window are:

\begin{itemize}

\item {\bf iwconfig}:  
Shows information about all your wireless connections.

\item {\bf ifconfig}
Shows information about all your network connections.

\item {\bf iwlist} (with the {\bf scanning} option):
Shows information about all detected wireless access points.

\item {\bf dmesg}:
Shows a record of your last bootup.  This may show error messages
regarding your WiFi card.

\end{itemize}

% For example, to select a particular wireless access point named
% X, type
% 
% \begin{Verbatim}[fontsize=\relsize{-2}]
% iwconfig wlan0 essid "X"
% \end{Verbatim}

% (Do this BEFORE running {\bf dhclient}.)

You can activate/deactivate your netword card during a session.  In
GNOME, this is done via System $|$ Administration $|$ Network.

\subsubsection{If You Have a Problem}

WiFi might work for you ``right out of the box,'' with no configuration
on your part.  If not, this section is for you.

Some wireless network cards typically sold with PCs today do not have
direct Linux drivers available.  A common example is the Broadcom
BCM43XX series.  However, you can still operate as usual after some
preparation:  

\textbf{Know Your WiFi Card}

You first need to determine which wireless card you have.  On the laptop
I use now, I determined this by running {\bf dmesg} and {\bf lspci}
under Linux, and by exploring under Windows.  Sure enough, it turned out
to be a Broadcom BCM43XX series card.

I then obtained the driver files for my Broadcom wireless card.  Windows
said that it was using {\bf bcmwl5.sys} for this card.  I got it from my
Windows partition, which on my machine is at

\begin{Verbatim}[fontsize=\relsize{-2}]
c:\windows\system32\drivers\bcmwl5.sys 
\end{Verbatim}

(As noted in Section \ref{access}, this may not be easy to do directly.
If you have trouble, boot Windows and copy the file to a USB key, then
go back to Linux and read from the key.)  Or, I could have downloaded it
from the Web.

\textbf{BCM43XX Series}

As mentioned, many laptops come with this card.  If your version of
Linux uses kernel 2.6.15 or newer, then things will be pretty easy, as
the kernel does include a driver for your card.

In Ubuntu, you merely need to request that the driver be downloaded and
installed, as follows:\footnote{This is for Ubuntu 8.04; 7.10 is very
similar)}  Select System $|$ Administration $|$ Hardware Drivers.  Check
the Enable box for Firmware for Broadcom 43 Wireless Driver.  You will
be asked whether you want the firmware to be downloaded from the net;
say yes.  Then check Enabled after the download.

\textbf{Other Cards/Kernels}

For other machines, go to the {\bf ndiswrapper} home
page, \url{http://ndiswrapper.sourceforge.net/}.  The program {\bf
ndiswrapper} allows Linux to use Windows drivers.

% Here is how I got my wireless card working under Fedora 3 and 4:
% 
% \begin{itemize}
% 
% \item You'll need to have the kernel sources installed.  The {\bf
% ndiswrapper} file {\bf INSTALL} suggests running
% 
% \begin{Verbatim}[fontsize=\relsize{-2}]
% ls /lib/modules/`uname -r`/build
% \end{Verbatim}
% 
% to check whether the proper files are there.\footnote{Note the reverse
% apostrophes, which run the command {\bf uname -r}, which returns the
% version of the kernel.}
% 
% If you don't have the files, you'll need to either get them from the CDs
% or DVD that you installed Linux from, or download them.  Either way, put
% them in the proper subdirectory of {\bf /usr/src} and then make a link,
% as described in the Wiki cited below.
% 
% \item I downloaded the source of {\bf ndiswrapper} from the home page,
% \url{http://ndiswrapper.sourceforge.net/}.  I then followed the
% instructions on installation at that page (click on Wiki $|$
% Installation).
% 
% \item I installed it, following the simple instructions.
% 
% (that's an ``ell,'' not a ``one,'' before the 5) and found that it was
% different from the one I had downloaded.  It later turned out that the
% downloaded one was wrong, though {\bf bcmwl5.inf} was all right. 
% 
% \item I put the various {\bf .sys} and {\bf .inf} files into the same directory,
% as requested.
% 
% \item I installed the driver, by typing 
% 
% \begin{Verbatim}[fontsize=\relsize{-2}]
% ndiswrapper -i bcmwl5.inf
% \end{Verbatim}
% 
% in that directory.
% 
% This created the directory {\bf /etc/ndiswrapper/bcmwl5}, with the {\bf
% .inf}, {\bf .sys} and other files there.\footnote{If I ever want to
% delete it, I will type
% 
% ndiswrapper -e bcmwl5
% 
% }
% 
% \item As a check that the driver was installed, I typed
% 
% \begin{Verbatim}[fontsize=\relsize{-2}]
% ndiswrapper -l
% \end{Verbatim}
% 
% \end{itemize}
% 
% That got everything installed.  Now, each time I boot up,\footnote{I can
% automate this if I wish.}, I will type
% 
% \begin{Verbatim}[fontsize=\relsize{-2}]
% modprobe ndiswrapper
% dhclient
% \end{Verbatim}
% 
% The first command automatically sets up a {\bf wlan0}, and the wireless
% card starts working, e.g. scanning for wireless access points.  The
% second command starts up the DHCP client, so that my machine will accept
% an IP address assigned by a wireless access point.

% \textbf{Kernel Problems}
% 
% As of September 2007, in using Fedora 7 I ran into troubles.  I was
% getting an error message ``SET failed on device wlan0:  operation not
% supported'' upon bootup, and WiFi wouldn't work.  In fact, the pilot
% light for my wireless card did not go on.  This was remedied by
% downloading the latest version of the kernel.  The error message still
% appeared, but I was able to use WiFi by running {\bf modprobe bcm43xx}
% by hand.

\textbf{Other Considerations}

I found in one wireless site that there seemed to be a problem with DNS,
the system that translates ``English'' addresses like {\bf
wwww.google.com} to their numerical counterparts, e.g.
{\bf 66.102.7.104}.  If you find that the former fails but the latter
works, you probably have a DNS problem.  

One way to handle this would be to configure your machine to have a
secondary DNS site.  You can use one given to you by your ISP, for
instance.  To add it, use the network configuration tool in your Linux
distro.  For example, under the GNOME GUI, select System $|$
Administration $|$ Networking $|$ DNS.

\subsubsection{Encryption}

For information on how to deal with WPA encryption, go to
\url{http://computerbits.wordpress.com/2006/10/27/fedora-core-6-installation-notes/}
or plug something like ``Fedora BCM43XX fwcutter'' into Google.

\subsection{Configuring KDE/GNOME for Convenient Window Operations}
\label{config}

\subsubsection{Autoraise Etc.}

You should find that windowing operations are generally easier in Linux
systems than in Windows, in the sense of requiring fewer mouse clicks,
if you set things up that way.  Personally, I find it annoying in
Windows that, when I switch from one window to another, I need to click
on that second window.  In most Linux windowing systems, I can arrange
things so that all I have to do is simply move the mouse to the second
window, without clicking on it.  The term for this is {\it focus follows
mouse}, and we can configure most Linux windowing systems to do this.

Also when I move from one window to another, I want the second one to
``come out of hiding'' and be fully exposed on the screen.  This is
called {\it autoraise}, and can be configured too.

You can arrange this configuration in less than one minute's time.
Again, the exact configuration steps will vary from GNOME to KDE, and
from one version to another within those systems, so I can't give you
the general steps here but here is how it works on a GNOME
system:  click System $|$ Preferences $|$ Windows, and check Select
Windows When the Mouse Moves Over Them (this may be referred to as {\it
focus} on your system) and Raise Selected Windows After an Interval
(this may be referred as {\it autoraise}).  I move the slider for the
latter all the way to the left, for 0.0 seconds.  For KDE, as of
September 2007 the sequence is K $|$ Control Center $|$ Desktop $|$
Window Behavior; after that, the choices are similar to those described
for GNOME above:  at Policy, choose Focus Follows Mouse and Auto Raise.

\subsubsection{Saving Window Places Between Sessions}

If upon bootup you'd like to have the same windows in the same places 
as in your last session, you can arrange this to occur automatically in
GNOME by System $|$ Preferences $|$  Sessions $|$ Session Options and
then checking the proper box.

\section{Some Points on Linux Usage}

To log out in GNOME, select System  $|$ Shutdown.  It is similar
for other desktop managers.

\subsection{More on Shells/Terminal Windows}
\label{shells}

In Microsoft Windows, most work done by most users is through a
Graphical User Interface (GUI), rather than in a command window (Start
$|$ Run $|$ cmd).  In Linux, a lot of work is done via GUIs but also it
is frequently handier to use a command window, called a {\it terminal}
window.  You should always keep two or three terminal windows on your
screen for various tasks that might arise.  

You can start a terminal window in GNOME by selecting Applications $|$
Accessories $|$ Terminal; the other desktop managers are similar.

You may be given a choice of several terminal types, say {\bf
gnome-term}, {\bf xterm} etc., but it doesn't much matter which one you
choose.\footnote{You may like {\bf gnome-term} because it is more easily
configurable, as to colors, size, etc.}  If you are using {\bf
gnome-term}, you may wish to reduce the font size, by holding down the
Control key and hitting the - key twice.

When you type commands in a terminal window, the program which reads and
acts on those commands is called a {\it shell}.  (Thus a terminal window
is sometimes called a ``shell window.'')

I have an introduction to Unix shells, based on the T C-shell, {\bf tcsh}
at \url{http://heather.cs.ucdavis.edu/~matloff/UnixAndC/Unix/ShellIntro.html}
and \url{http://heather.cs.ucdavis.edu/~matloff/UnixAndC/Unix/CShellII.html}.

The default shell in Linux is {\bf bash}. It is very good, but if you
are used to using, say, {\bf tcsh},\footnote{Or if you want to use my
shell tutorials, mentioned above.} you can use the {\bf chsh} command in
any terminal window to change your login shell.

\subsection{Cut-and-Paste Window Operations}

The X11 windowing system used in Linux has its roots in 3-button mice.
Today, most people have such mice (the middle wheel counts as a button),
but if you don't, that's no problem, because Linux does 3-button
emulation for you.  The middle button is emulated by simultaneously
clicking both left and right buttons.

To do a cut-and-paste operations, hold down the left mouse button and
drag it to highlight the text you wish to copy.  Then go to the place
you wish to copy that text, and simultaneously push both the left and
right buttons.  Generally, more things are cut-and-pastable in Linux
than Windows, so this is a big convenience.

\subsection{Mounting Other Peripheral Devices}

This section explains how to use DVDs, USB devices and so on under
Linux.

\subsubsection{Mount Points}

Each I/O device that contains a file system must be {\it mounted}, i.e.
associated with some directory.  That directory is called a {\it mount
point}.  The files then appear in that directory.

These days most Linux distributions have a designated directory for
mount points for DVD/CD-ROMs, USB devices, floppy disks, etc.  This will
vary from one distribution to another, but typical directory names are
{\bf /mnt}, {\bf /media} etc.

You can check what is currently mounted by running the {\bf df} command
from a shell window (another good Linux learning experience).  The mount
points are listed along with the {\bf /dev} files.\footnote{Recall from
Section \ref{whatispart} that every I/O device is viewed by Linux as
some {\bf /dev} file.}  Also, to list the {\bf /dev} files for all your
operating drives including USB flash drives, type {\bf fdisk -l}.

For more detailed information, such as file system types, just run {\bf
mount} without any arguments.

Your machine's internal hard drives, and possibly other devices, will be
mounted automatically at boot time.  Many, but not all, such devices and
their mount directories is maintained in the file {\bf /etc/fstab}.  The
details are an advanced topic, but even without understanding
everything, you might find it worthwhile to take a quick look at that
file.  

When you attach a device to your machine {\it after} bootup, your system
will probably recognize it immediately, and maybe pop up a window
showing the device's contents.  If you have trouble, you can use the
Unix {\bf mount} command.  This is an advanced command, but just to give
you an idea, a typical usage would be

\begin{Verbatim}[fontsize=\relsize{-2}]
mount -t iso9660 /dev/hdc /mnt/yyy
\end{Verbatim}

This tells Linux that the I/O device corresponding to {\bf /dev/hdc},
our CD-ROM, should be mounted at the directory {\bf /mnt/yyy}.
If that directory doesn't exist, you must create it first, using {\bf
mkdir}.  The field {\bf -t iso9660} says that the file system type is
ISO9660.  This is standard for CD-ROMs, and you can probably omit it.

\subsubsection{Reading Your DVD/CD-ROM and Floppy Drive from Linux}
\label{dvd}

The files are available under the mount point, as explained above.
If they contain music or video, you of course will need a program to
access them; see Section \ref{mplayer}.

\subsubsection{CD/DVD Burning}

You can use the shell-based {\bf cdrecord} and {\bf dvdrecord} programs,
but it is much easier to use one of the GUI-based programs.  I use {\bf
gnomebaker}.

If you do not have that program, you can download it from the Web.
Under Ubunta, for instance, simply type

\begin{Verbatim}[fontsize=\relsize{-2}]
sudo apt-get install gnomebaker
\end{Verbatim}

Run the program by typing

\begin{Verbatim}[fontsize=\relsize{-2}]
gnomebaker
\end{Verbatim}

in a shell window.  The GUI will come up.  

In the bottom right-hand corner, set the size of the CD/DVD (a typical
DVD has capacity 4.7G), then click Create Data Disk.

Then go to the Filesystem section in the upper-right portion of the
window, and choose your directory.  Then for each file you want to burn,
click and drag it from the File section at the upper-right to the Data
Disk (or Audio Disk) section at the bottom of the window.  If you wish
to copy an entire directory, just drag the directory name.

To burn an ISO image, select Actions $|$ Burn CD/DVD Image, then 
select the {\bf .iso} file, and burn.

% I prefer to use the plain {\bf cdrecord} command, included with most
% Linux distributions.
% 
% To burn a data CD, you'll need it to be in ISO form.  In some cases,
% e.g. where you download material from the Web which you want to burn
% into a CD, you may already have an ISO file, i.e. with {\bf .iso}
% suffix in its name.  
% 
% If not, then first put all your data in some directory, say {\bf x}.
% Then make an ISO file from it, say named {\bf y.iso}:
% 
% \begin{Verbatim}[fontsize=\relsize{-2}]
% mkisofs -r -o y.iso x
% \end{Verbatim}
% 
% Put a blank CD in the tray.  (Do NOT mount it, as it has no file system
% to mount.) Then log in as root. 
% 
% Your CD device number might not be 0,0 (or 0,0,0; the first 0 may be
% omitted in certain cases).  Type
% 
% \begin{Verbatim}[fontsize=\relsize{-2}]
% cdrecord -scanbus
% \end{Verbatim}
% 
% to find out.
% 
% Now burn the file:
% 
% \begin{Verbatim}[fontsize=\relsize{-2}]
% cdrecord -eject -v -isosize speed=2 dev=0,0 y.iso
% \end{Verbatim}
% 
% A speed of 2 is very conservative, maximizing the chance that the burn
% has no errors.  If you wish, try omitting the {\tt speed} field in your
% command, and {\bf cdrecord} may choose a higher speed.
% 
% A similar {\bf dvdrecord} command exists for burning DVDs.

\subsubsection{Using USB Devices}

USB drives, including memory sticks, should have their filesystems
mounted automatically when you attach them.  Use the {\bf df} command to
check where they've been mounted (it could be in the directory {\bf
/mnt/} {\bf /media} etc.). 

USB mice should become automatically usable when you attach them.

\subsection{A Note on Ubuntu}

\subsubsection{Root Operations}
\label{sudo}

Ubuntu works like any other Linux distro, except for one important
point:  Ubuntu does not have a root user account in the classic Unix
sense.  Instead, whenever executing a command which requires root
privileges, one precedes the command by the term {\bf sudo} (``superuser
do'').  One is then prompted for a password, which is the password for
the first user account created at the time of installation.

If you have a lot of root-type work to do in a session, type

\begin{Verbatim}[fontsize=\relsize{-2}]
$ sudo -s
\end{Verbatim}

to create a new superuser shell, and do your work there.

\section{Linux Applications Software}

\subsection{GUI Vs. Text-Based}

Most people prefer to use GUI-based applications.  If you are one of
them, rest assured that there are tons of them available for Linux.

I do wish to mention, though, that the ``super hard core'' Linux users
prefer to use text-based applications, rather than GUI ones.  For
instance, I and many others like the {\bf mutt} e-mail utility (Section
\ref{email}), which is text-based.  Here's why, at least in my view:

\begin{itemize}

\item I often access my Linux machine remotely, while
traveling.\footnote{Which is in fact exactly the case as I write this
paragraph.}  I might be at a university library, for instance, or at the
business center in a hotel, and be ``stuck'' with a Windows machine, and
logging in to my Linux machine via an SSH connection.\footnote{Though I
sometimes use VNC to access a remote image of my Linux desktop.  See
\url{http://heather.cs.ucdavis.edu/~matloff/vnc.html}.}  This limits me
to text.

\item It's very important to me that I use the same text editor for all
my computer applications---e-mail, programming, word processing,
etc.---so that I can take advantage of all the abbreviations, shortcuts
and so on which I have built up over the years.  This saves me huge
amounts of typing.  But most GUI applications, e.g. e-mail utilities,
have their own built-in text editors, so I can't use mine.

\item I find that text-based applications often have more features, are
better documented, etc.  For example, I often wish to automate certain
processes, such as uploading files to another machine, and typically
text-based programs do this better.

\end{itemize}

However, in listing my favorite applications in Section \ref{favorites}
below, I've made sure to list both text-based and GUI programs.

\subsection{My Favorite Unix/Linux Utilities and Applications} 
\label{favorites}

\subsubsection{Text Editing}

I use a modern extension to the {\bf vi} editor, {\bf vim}.  This is
the version of {\bf vi} which is built in to most Linux distros.  See my
tutorial at \url{http://heather.cs.ucdavis.edu/~matloff/vim.html}.   

Note:  In the Fedora distro, somehow the version of {\bf vim} that is
linked to {\bf vi} isn't configured fully correctly.  I suggest using
{\bf /usr/bin/vim} directly.

Even though {\bf vim} is text-based, it does have a GUI version too,
{\bf gvim}.  This comes with nice icons, allows you to do mouse
operations, etc.  Unfortunately, most Linux distros seem to have only
the text-based program.  To get the GUI, you can download it yourself.
In Ubuntu, do

\begin{Verbatim}[fontsize=\relsize{-2}]
sudo apt-get install vim-gnome
\end{Verbatim}

For this, you may need to edit {\bf /etc/apt/sources.list} and
uncommented the lines for Canonical's 'partner' repository

\subsubsection{Web Browsing}
\label{yessometimesiuselynx}

Your Linux distro will come with a Web browser, probably Firefox, and
possibly Konqueror in addition.

I almost always use Firefox.  But believe it or not, sometimes I use the
famous text-based browser, {\bf lynx}.  In some cases, it is just plain
quicker and easier.  Moreover, you can do cool tricks, such as recording
keystrokes for later playback, thus enabling one to do certain Web
operations automatically.

If you use Fedora, your Firefox system may not be configured for Java.
If so, see
\url{http://www.mjmwired.net/resources/mjm-fedora-fc6.html#java}.  NOTE
CAREFULLY:  This site has some very long shell commands, which will not
be completely displayed unless you make the browser window quite wide.

If you are short on memory (i.e. RAM), you may wish to use a lightweight
browser, such as Galeon (related to Firefox but somewhat fewer features)
or Dillo (really bare-bones).

\subsubsection{E-Mail}
\label{email}

I use the {\bf mutt} e-mail utility. It is very flexible and
customizable, and excellent features.  For example, it has great search
capabilities, important if you are a heavy e-mail user.  I like its
ability to record the fact that one has already replied to a message,
and the fact that it allows you to save partially-written message for a
later time when you can finish writing it.  It is text-based, not GUI,
but the functionality it gives is what really counts, in my view.  See
my tutorial at \url{http://heather.cs.ucdavis.edu/~matloff/mutt.html}. 

In Ubuntu, download it by typing

\begin{Verbatim}[fontsize=\relsize{-2}]
sudo apt-get install mutt
\end{Verbatim}

If you prefer a GUI-based mail utility, many nice ones exist for Linux.
Check the Web for these, or use the Thunderbird e-mail utility in the
Firefox Web browser suite.
 
\subsubsection{HTML Editing}

I usually use Vim, along with some macros I've written for HTML editing,
but I sometimes use Amaya, which is a full-featured GUI HTML editor,
written by the Web policy consortium.  One nice feature is that you can
actually use the embedded Web links, good for testing them.  See my
tutorial at \url{http://heather.cs.ucdavis.edu/~matloff/amaya.html}. 

There are many newer and more powerful packages, such as Quanta+,
Bluefish and NVu.

\subsubsection{Integrated Software Development (IDE)}

For C/C++ work, I actually don't use an IDE.  I find that the {\bf vim}
editor (cited above) and the {\bf ddd} GUI interface to the {\bf gdb}
debugging tool, work great together.  For example in {\bf vim} I can
type {\tt :make} (which I have aliased to just {\tt M}, or with {\bf
gvim} click on the make icon, and the source code I'm debugging will be
recompiled.  And as I've mentioned, it's important to me that I use the
same text editor for all applications, which most IDE would not allow me
to do.  I use either GDB (try CGDB!) or DDD for my debugging tool.  See
my tutorials at \url{http://heather.cs.ucdavis.edu/~matloff/vim.html}
and \url{http://heather.cs.ucdavis.edu/~matloff/debug.html}. 

DDD is also usable with my favorite programming language, Python.

However, if you love IDEs, try Eclipse.  I've got a tutorial that is
more complete than most, at
\url{http://heather.cs.ucdavis.edu/~matloff/eclipse.html}.   It can be
used with C, C++, Java, Perl, Python and many others.

Also, for KDE users, there is a very well-received IDE named KDevelop.
I lean toward Eclipse, though, as it is easier to learn, is
cross-platform, and can be used with more programming languages. 

\subsubsection{Word Processing}

I use \LaTeX\, because of its flexibility, its beautiful output, and its
outstanding ability to do math. You may like Lyx, which is a great
GUI interface to \LaTeX\ which is especially good for math work.
See my tutorials at
\url{http://heather.cs.ucdavis.edu/~matloff/latex.html} and
\url{http://heather.cs.ucdavis.edu/~matloff/lyx.html}. 

If you wish to work with files compatible with the Microsoft Office
environment, there is a free suite of programs, OpenOffice, which
provide Microsoft compatibility.  It is packaged with most Linux
distributions.

If you would like something that quickly converts an Office file to
rough text form, say to use with e-mail attacments, try Antiword.
In Ubuntu, install via

\begin{Verbatim}[fontsize=\relsize{-2}]
sudo apt-get install antiword 
\end{Verbatim}

\subsubsection{Playing Movies, Music, Etc.}
\label{mplayer}

MPlayer is free and very good.  Its capabilities are amazingly broad.

The documentation is extensive, and hard to navigate, but here are a
couple of things to get you started:

{\bf Installation:}

It's easy in Ubuntu:

\begin{Verbatim}[fontsize=\relsize{-2}]
sudo apt-get install mplayer 
sudo apt-get install mencoder 
\end{Verbatim}

Otherwise, build it yourself, as follows.

One downloads the source code, {\bf MPlayer-1.0pre7try2.tar.bz2} and the
codecs, {\bf essential-20041107.tar.bz2}, from
\url{www.mplayerhq.hu/design7/dload.html}.  

Unpack the codecs file first, 

\begin{Verbatim}[fontsize=\relsize{-2}]
tar xfj essential-20041107.tar.bz2
\end{Verbatim}

This creates a new directory.  Copy the contents of that directory to
the directory {\bf /usr/local/lib/codecs} (use {\bf mkdir} to create it
if necessary).  (Note:  There may be legality issues with some codecs.
When in doubt about a particular codec, you should obtain it from a site
like Fluendo that offers it for a nominal fee,  See a discussion at
\url{http://fedoraproject.org/wiki/CodecBuddy}.

Now, unpack the source code file, and go into the directory it creates.
Then go through the usual sequence for building open-source software
from source:

\begin{verbatim}
configure
make
make install
\end{verbatim}

Note that if you want to use the GUI, the {\bf configure} command should
be

\begin{Verbatim}[fontsize=\relsize{-2}]
configure --enable-gui
\end{Verbatim}

After {\bf make install} is done, you will probably get a message
something like

\begin{Verbatim}[fontsize=\relsize{-2}]
 *** Download font at http://www.mplayerhq.hu/dload.html
*** for OSD/Subtitles support and extract to
/usr/local/share/mplayer/font/
*** Download skin(s) at http://www.mplayerhq.hu/dload.html
*** for GUI, and extract to /usr/local/share/mplayer/skins/
\end{Verbatim}

The fonts are needed for the subtitles (and for the GUI, if you use it).
Just the {\bf iso1} font is needed.  Download the font package, go to
the indicated directory ({\bf /usr/local/share/mplayer/font/} in the
above example), and then do the unpack operation.  This will produce a
subdirectory, e.g. {\bf font-arial-iso-8859-1}.

{\bf Viewing a video:}

To play a video or audio file, say {\bf x.avi}, type

\begin{Verbatim}[fontsize=\relsize{-2}]
mplayer x.avi
\end{Verbatim}

If you specify several files, as a playlist, it will play them all.  Hit
the Enter key if you want to skip the rest of the current file and go to
the next one.

To play a DVD, put the disk in the tray (see Section \ref{dvd}).  Then
type

\begin{Verbatim}[fontsize=\relsize{-2}]
mplayer dvd://1 -dvd-device /mnt/cdrom
\end{Verbatim}

where you will have to substitute a different mount point if it is not
{\bf /mnt/cdrom} (try running {\bf df} or rummaging around in {\bf
/media}).

You have the following controls:

\begin{itemize}

\item right and left arrow keys to go back or forward 10 seconds

\item down and up arrow keys to go back or forward 1 minutes

\item PgDown and PgUp keys to go back or forward 10 min

\item left- and right-bracket keys to decrease/increase speed by 10\%,
or left- and right-brace for 50\%; Backspace key to return to normal
speed

\item Space bar to pause, then . to go forward frame by frame, Space bar
to resume play

\item f to go full screen

\item q to quit

\end{itemize}

You can use {\bf mplayer}, actually {\bf mencoder}, which comes with the
package, to do format conversion, e.g. AVI to MPG, change aspect ratio, 
and even do some primitive editing.

There are many, MANY, {\Large \bf MANY} different options.

You may wish to try other players, e.g. VLC.

\subsubsection{Video Editing}

Try Kino, Cinelerra, LiVES and many others.

\subsubsection{Image Viewing, Manipulation and Drawing}

I use {\bf xpdf} to view PDF files, though Acroread for Linux is
available.  I like the fact that {\bf xpdf} allows me to copy ASCII text
from the file.

For collections of JPEG files and the like, I use {\bf xzgv}; for
viewing a single image, I use {\bf qiv}.

Want something like Adobe Photoshop?  The GIMP program is quite
powerful, and free.  It's included with most Linux distributions.

You can use GIMP to draw, but for ``quick and dirty'' tasks, I would
suggest Dia, at \url{http://www.gnome.org/projects/dia/}.

\subsubsection{Accessing Usenet Newsgroups}

Linux distros generally come a text-based newsreader, either {\bf slrn}
or {\bf tin}.  I generally use {\bf slrn}, but am not that happy with
any known newsreader.

In the GUI arena, I sometimes use {\bf pan}.  You can download it from
\url{pan.rebelbase.com}.  

Firefox's Thunderbird program includes a newsreader too.

\subsubsection{FTP}

I usually use the text-based {\bf ftp} and {\bf sftp}, the latter being
an SSH version for security.  

If you do frequent uploads/downloads to/from a particular site and wish
to automate them, another text-based program, {\bf yafc}, is excellent.

A very nice GUI program, though, is {\bf gftp}, which you can download
from the Web if your Linux system doesn't already have it.  In addition
to the GUI, this program also has some functionality which ordinary FTP
programs don't have.

\subsubsection{Statistical Analysis}

Use the statistical package that the professional statisticians use---R!

In my opinion from the point of view of someone with a ``foot in both
camps''---I'm a computer science professor who used to be a statistics
professor---the R statistical package is the best one around, whether
open source or commercial.\footnote{In some respects, it's even better
than S, the commercial product it is based on.} It is statistically
modern and correct, and it also is a general-purpose programming
language.  

I have a tutorial on R at \url{http://heather.cs.ucdavis.edu/~matloff/r.html}.

\subsection{Downloading New Software}

There is a vast wealth of free software for Linux on the Web.  Here's
how to obtain and install it.

\subsubsection{How to Find It}

These days most downloads and installs are done automatically, say with
{\bf yum} or {\bf apt-get}, as seen in Section \ref{yum} below.  That
helps you find it too.  If you want to find application Z, instead of
plugging ``Z'' into Google, plug ``yum install Z'' or ``apt-get install
Z'' so as to narrow down the volume of response.

\subsubsection{Automatic Download/Installation}
\label{yum}

In recent years, most Linux distros have made it very easy to download
and install new software.  In Fedora, for instance, one uses the {\bf
yum} command.

For example, to download the program {\bf yafc} mentioned above, one
simply types

\begin{verbatim}
yum install yafc
\end{verbatim}

In Ubuntu, there is the {\bf apt-get} command, which works similarly.
For instance, to download the {\bf xpdf} PDF viewer, I typed

\begin{Verbatim}[fontsize=\relsize{-2}]
sudo apt-get install xpdf
\end{Verbatim}

(Note:  Ubuntu may ask you to install from your CD-ROM, but yours may be
incomplete.  If so, comment out the first line of {\bf
/etc/apt/sources.list}; this is the line telling Ubuntu to install from
the CD-ROM.)

(See Section \ref{sudo} for an explanation of {\bf sudo}.)

With both {\bf yum} and {\bf apt-get}, one can direct where to download
from, by making the proper entries in the file {\bf
etc/apt/sources.list}.  For instance, for the R statistical package
above, {\bf apt-get} may not find it on its own, in which case we can
add a line

\begin{Verbatim}[fontsize=\relsize{-2}]
deb http://cran.stat.ucla.edu/bin/linux/ubuntu gutsy/ 
\end{Verbatim}

to {\bf etc/apt/sources.list}, telling {\bf apt-get} that here is an
alternative place it can look.  (This is for the Gutsy edition of
Ubuntu.)

By default {\bf apt-get} will try to retrieve your requested program
from your installation CD/DVD.  You can change this by commenting-out
the line in {\bf etc/apt/sources.list} that begins with

\begin{Verbatim}[fontsize=\relsize{-2}]
deb cdrom:
\end{Verbatim}

Sometimes it may not be clear which package name to use with {\bf yum}
or {\bf apt-get}.  For instance, to install the GCC compiler, C library
and so on, the command is

\begin{Verbatim}[fontsize=\relsize{-2}]
sudo apt-get install build-essential
\end{Verbatim}

How did I learn this?  I did a Web search for ``apt-get GCC.''  

To install the {\bf curses} library (and include file), do

\begin{Verbatim}[fontsize=\relsize{-2}]
sudo apt-get install libncurses5-dev
\end{Verbatim}

\subsubsection{Using RPMs}

Though the methods in Section \ref{yum} have now made RPMs less
important, you may find that the software you want comes in an RPM
package, with a {\bf .rpm} suffix in its name.  To install such a
package, type

\begin{verbatim}

rpm -i package_file_name
\end{verbatim}

If you later wish to remove, i.e. uninstall a package, you can use {\bf
rpm -e} (`e' stands for ``erase''). You do NOT have to have the RPM file
present to do this.

Some packages will have different versions for different C libraries.
Red Hat uses {\bf glibc}. Type

\begin{verbatim}
ls -l /lib/libc*
\end{verbatim}

to see which version you have.

You may find that you need some library files for a program you
download, and that you are missing those files. You can usually get
these from the Web too. If a program complains about a missing file, try
the {\bf ldd} command (e.g. {\bf ldd x} if the name of the program which
needs the library is {\bf x}); this will tell you which libraries are
needed, where they were found on your system, and which ones, if any,
were not found.

\section{Dual-Boot Issues}

You may wish to change some parameters of your dual-boot process, e.g.
change the default OS.  You can do this by editing the configuration
file for your bootloader.

Most distros today use GRUB as their bootloader.  Its configuration
file is {\bf /boot/grub/menu.lst}.  By the way, note that GRUB's
notation for partitions is (drive ID, partition number), so that for
instance (hd0,1) means the second partition in the first hard drive.

\section{Learning More About Linux}

\subsection{Wanna Get Good at Linux?  Use It for Everything!}  
\label{useit}

The only way to really learn Linux is to use it on a daily basis for all
your computer work---e-mail, word processing, Web work, programming, etc.

As you do this, the expertise you'll want to pick up includes:  file,
directory and mount operations; process operations; roles of system
directories ({\bf /usr}, {\bf /etc}, {\bf /dev}, {\bf /sbin} and their
various subdirectories, e.g. {\bf /usr/lib}; search paths; network
operation and utilities such as {\bf netstat}; and so on.  

Don't try to do this all at once.  Instead, take your time, and learn
these naturally, as the need arises.  As you use Linux more and more in
your daily computer application work (e-mail, word processing, etc.),
the needs will arise as you go along.

And remember, there's lots of help available if you need it.

\subsection{Getting Help}
\label{help}

\subsubsection{Newsgroups}

There are various Usenet newsgroups devoted to Linux, a few of which
are:

\begin{verbatim}

comp.os.linux.setup
comp.os.linux.hardware
comp.os.linux.answers
comp.os.linux.announce (excellent for news of new programs, mostly
                     free, that run under Linux)
\end{verbatim}

By the way, if you have a problem with hardware and post a query about
it to a newsgroup, it is a good idea to include the output from the
{\bf dmesg} command. It gives a record of what occurred during bootup.

\subsubsection{The Web}

\begin{itemize}

\item If you are running Ubuntu or one of its offshoots, the Ubuntu
Forums, \url{http://ubuntuforums.org/} is an excellent resource.

\item Linux home page, at http://www.linux.org/ Lots and lots of
information is available here.

\item \url{www.linux.com}.  Chock full of information and links.

\item Google's excellent set of links to various Linux sites,
\url{http://directory.google.com/Top/Computers/Software/Operating_Systems/Linux}

\item Another good set of Linux links, 
\url{http://www.linuxjunior.org/resources.shtml}

\item If you are having trouble with specific hardware in your Linux
installation, an excellent place to go for detailed information is the
Linux HOW-TO documentation. (For the same reason, if you are about to
purchase a machine and suspect that some of the hardware is nonstandard,
you can check the corresponding Linux HOW-TO to see if there are any
problems with that hardware.

The HOW-TO documents are available at many sites, such as the one at
linux.org.

\end{itemize}

\subsubsection{LUGs}

There are Linux Users Groups (LUGs) in virtually every city.  You can
join if you wish, or just get to know them casually.  They are great
sources of help!  And by the way, many of them hold monthly Linux
Installfests, where you can see Linux being installed or have it
installed on your own machine.  

\section{Troubleshooting}

One of Linux's biggest strengths is its stability.  If you are tired of
getting Windows' infamous ``blue screen of death,'' then Linux is the OS
for you.  (It is also subject to far fewer virus and other attacks than
Windows.)  So emergencies are rare, but they can happen.  Here are some
tips for such cases.

\subsection{Tools}

Here are some commands you can run in a terminal window that you can use
to investigate:

\begin{itemize}

\item {\bf ps}:  Tells you what processes are running.  Typically
one uses this with something like the {\bf ax} option.

\item {\bf dmesg}:  Tells you the major events that have occurred
on your machine ever since it was last booted up.

\item {\bf lsmod}:  This tells you what OS modules are installed, i.e.
device drivers and the like.

\item {\bf lpq}:  Lists the current printer queue.

\item {\bf lsusb}:  Lists what USB devices are currently plugged in.

\item {\bf ifconfig}:  Lists network interfaces.

\item {\bf iwconfig}:  Lists currently operating wireless devices.

\item {\bf iwlist}:  Lists wireless access points in range.

\item {\bf netstat}:  Lists current network connections.

\end{itemize}

\subsection{A Program Freezes}

If an application program freezes up and you invoked it from the command
line within a shell, you can in most cases kill it by hitting Ctrl-c in
the terminal window from which invoked it.  If this doesn't work, run
the ``processes'' command by typing

\begin{Verbatim}[fontsize=\relsize{-2}]
ps ax
\end{Verbatim}

in another terminal window, and noting the process number of your
program.  Say for concreteness that that number is 2398.  Then type

\begin{Verbatim}[fontsize=\relsize{-2}]
kill -9 2398
\end{Verbatim}

to kill the program.

If you have a program named, say, {\bf xyz}, the command

\begin{Verbatim}[fontsize=\relsize{-2}]
pkill -9 xyz
\end{Verbatim}

kills all running instances of the program.

\subsection{Screen Freezes}
What if your entire screen freezes up?  Again, this should be quite
rare, but it is possible.  I recommend the following remedies, in order:

\begin{itemize}

\item Hit Alt F2, which will bring up a little window in which you can
run a command, say {\bf pkill} as above.  

\item Try going to another screen!  Linux allows you to switch among
multiple screens.  You can switch to the second screen via Alt F2 or
Ctrl 2, depending on your window manager.  Then open a terminal window in the
new screen, find the process number of the program and kill the program,
as described above.

\item Try hitting Ctrl Alt Backspace (all keys simultaneously).  This
should cause an exit from Linux's X11 windowing system but not an exit
from Linux itself.  You would then get an opportunity to log in again.

\end{itemize}

Try NOT to simply poweroff the machine, as that may do damage to your files.
It may not be permanent damage, as the OS will try to fix the problems
when you next reboot, but don't just pull the plug unless you have no
other recourse.

\subsection{Inaccessible Partition}

Suppose you reinstall or upgrade your Windows OS.  This will probably
restore the original boot procedure, rendering your Linux files
inaccessible.

You can easily access the files by booting one of the live CD distros
(Section \ref{live} above).  Do the following after booting:

\begin{Verbatim}[fontsize=\relsize{-2}]
$ cd /
$ mkdir mylinfiles
$ mount /dev/hda2 mylinfiles
$ cd mylinfiles
$ ls
\end{Verbatim}

(Of course, you may need to type a different {\bf /dev} file name here;
see Section \ref{whatispart} above.)

At this point, you will be in your Linux file system!  You can then go
down to your Linux home directory, via {\bf cd home} or something like
that.

You can then run GRUB from your live CD.  Please check the Web for
instructions.

\section{If You Are Upgrading or Replacing Another Version or
Distribution of Linux}
\label{upgrade} 

(If you are installing Linux from scratch, skip this section.) 

Suppose you already have Linux installed but are upgrading to a newer
version of the same distribution or changing to a different
distribution.  First of course you will want to make sure you back up your
old files, just in case sometimes goes wrong. 

Note that in addition to any ``personal'' files you have, you may also
have added some downloaded packages, whose files are now in places like
{\bf /usr/local/}.  You may also have modified files in {\bf /etc}, such as
{\bf /etc/resolv.conf}.  You may wish to tar these into a save file too.
(Don't copy the Linux system files, e.g in {\bf /usr/bin}, though, since you
want them to be replaced by their counterparts in the new version of
Linux.)

% \section{Making an MS-DOS Boot Disk}
% \label{making}
% 
% Insert a 3.5" floppy disk. Go to My Computer, then Control Panel, then
% Add/Remove Programs, then Startup Disk.
% 
% \section{Winmodems}
% 
% Currently virtually all mass-produced PCs and laptops use {\it
% Winmodems}.  These are only ``partial'' modems, in the sense that the
% hardware is incomplete.  The missing hardware is replaced by software --
% Windows software, which presents a problem for Linux users.
% 
% Here are your choices for solving the problem:
% 
% \begin{itemize}
% 
% \item Install another modem. Check the Linux modem Web page, at
% http://www.grapevine.net/~gromitkc/winmodem.html before you buy.  Or you
% may wish to avoid the problem entirely by purchasing an external modem
% which connects to your serial port; most external modems are fine,
% according to the modem Web page above. (You can probably even leave your
% old internal modem inside the machine, as long as you unplug the phone
% wire from it.) Plug your modem into your serial port, and rerun {\bf
% modemtool}.
% 
% \item Try to find a Linux driver for your Winmodem.  There is a lot of
% activity in this regard; see www.linmodems.org
% 
% \item Or do all your Internet access from Windows, transferring your
% downloaded/uploaded files to and from Linux via Windows. To do this, see
% page \ref{access}.
% 
% \end{itemize}
% 
% \section{Who Does NOT Need to Run fips}
% \label{nofips}
% 
% The main types of people who do NOT need to run {\bf fips} are:
% 
% \begin{itemize}
% 
% 
% \item Those who have already have two or more partitions on their 
%     hard drive. 
% 
% \item Those who have two hard drives, and will be installing Linux on
%     the second one. (Make sure you have formatted the second drive
%     using Windows, even though you will be installing Linux there.)
% 
% \item Those who have a new machine and possess the CD-ROMs with all its
%     software, and are willing to rebuild their Windows system from
%     scratch from those CD-ROMs. In this case, you would re-format C:
%     then run DOS' (not Linux's) fdisk program to make two partitions
%     on C:, then reload all your Windows software onto the first
%     partition.
% 
% \end{itemize}
% 
% In these cases, skip the {\bf fips} material in this Web document, and go
% directly to the main Linux installation section.
% 
% \section{How to Change LILO Parameters}
% \label{changelilo}
% 
% You can change LILO's parameters by editing LILO's startup file,
% /etc/lilo.conf and then running /sbin/lilo without command-line
% arguments. (The changes to the startup file won't take effect unless
% you then run lilo.)
% 
% For example, LILO has a parameter named ``timeout", which specifies the
% number of tenths of seconds LILO will wait for your response after it
% issues its ``LILO boot:" prompt. For instance, if you edit the timeout
% line in the startup file to read
% 
% \begin{verbatim}
% 
% timeout=150
% \end{verbatim}
% 
% then your default OS will automatically be booted if we don't type
% anything within 15 seconds of the appearance of the ``Lilo boot:"
% prompt.
% 
% The installation process sets Linux as the default OS. If you wish to
% change this to Windows (called ``dos" in LILO), use your editor to
% change the ``default" line in /etc/lilo.conf

\section{Accessing Your Windows Files from Linux}  
\label{access}

At this point, most Linux distributions, except Fedora/Red Hat, give you
access (at least read access) to your Windows partition from Linux.  For
some of them, they may do this automatically, in which case your Windows
partition, say {\bf /dev/hda1} should be visible in the file {\bf
/etc/fstab}.  If not, mount it yourself:  

\begin{Verbatim}[fontsize=\relsize{-2}]
mkdir /dosc
mount /dev/hda1 /dosc
cd /dosc
\end{Verbatim}

You should now see your Windows files, and should be able to access them
on at least a read basis.

For more information, including concerning write access, ss the
Linux-NTFS Project, \url{http://www.linux-ntfs.org/}.

% First, determine the exact kernel your Linux system has, say by running
% the {\bf dmesg} command, mentioned in the first few lines of the output.
% Then go to the above Web page, click on your distribution, say Fedora 3,
% and download the RPM with the same kernel number.  Then follow the
% directions given at the Web page, running {\bf rpm} to install the
% Linux-NTFS package, and then running {\bf modprobe} and {\bf mount}.  
% 
% All of that is the first time.  On subsequent when you wish to access
% your Windows partition, you need run only the {\bf mount}.

\section{If You Wish to Remove Linux}

If you wish to remove Linux from your machine, first remove LILO/GRUB as
follows.  Boot from your the Windows recovery CD that came with your
machine.  (Make sure you have the boot order set for your machine so
that it tries to boot from CD or DVD before a hard drive.)
When asked whether you want setup or recovery, hit R for the
latter.  Choose whichever disk your Windows system is on, probably C:.
Change directories to WINDOWS if you are not already there, and issue
the FIXMBR command.  It will warn you that you will be restoring the
Master Boot Record (MBR), which is what you want.  Then hit EXIT to
finish, and reboot without the CD.

Subsequently Windows will boot up as it did before you installed
Linux.  

Finally, use GParted to recover the former Linux space into your Windows
partitions.  Typically, this means deleting your Linux partitions (the
ones that are not of type FAT32 or NTFS), and then expanding your NTFS
partition.  Don't forget that the next time you boot Windows, it will
ask you if you want a disk check, which you should definitely answer Yes
to.

% \section{How to Use wvdial}
% \label{wvdial}
% 
% The {\bf wvdial} program does a great job of automating the setup of a PPP
% connection to the Internet.  It is already there in your Red Hat
% distribution.
% 
% Create a configuration file by typing
% 
% \begin{verbatim}
% 
% wvdialconf /etc/wvdial.conf
% \end{verbatim}
% 
% or by simply modifying the example in the {\bf wvdial} man page.
% 
% You are now ready to try it out! Just type
% 
% \begin{verbatim}
% 
% wvdial &
% \end{verbatim}
% 
% The program should dial out for you, sense when the user and password
% prompts come from your ISP, automatically send them, and then start {\bf
% pppd}.  You now can run {\bf telnet}, {\bf netscape} or whatever.
% 
% To quit your PPP connection, first bring {\bf wvdial} back to the foreground
% by typing ``fg", and then hit ctrl-c to kill it.
% 
% (Do not use the Unix {\bf kill} command for this; ctrl-c gives {\bf wvdial} a
% chance to end gracefully. Otherwise, the OS kernel's routing table
% will not revert to its original setting, and if you use {\bf wvdial} again
% without rebooting, {\bf wvdial} won't work properly. If you are curious
% about this, try running the {\bf route} command before, during and after
% running {\bf wvdial}. 
% 
% \section{Making a Linux Rescue Disk}
% 
% If you ever have a disk crash or similar problem, you'll be thankful
% that you had previously made a Linux rescue disk.  If you did nto do so
% during your Linux install process, put a floppy diskette into the drive
% and run the {\bf mkbootdisk} command.
% 
% \section{Dial-Up Networking from Linux}
% 
% Important note: Beware of ``Winmodems,'' which will not work with Linux.
% See our section on this elsewhere in this document, titled
% ``Winmodems.''
% 
% To use your modem (if not a Winmodem), first find out which COM port
% your modem is on. (See the material above on how to determine your
% hardware properties in Windows.) Then run the Linux program {\bf
% modemtool}.
% 
% Then edit the file {\bf /etc/resolv.conf} (create it if it is not there),
% adding the line
% 
% \begin{verbatim}
% nameserver x.x.x.x
% \end{verbatim}
% 
% to it, where x.x.x.x is the DNS address used by your Internet service
% provider. (You can have up to three such lines.)
% 
% I strongly recommend using {\bf wvdial}, It probably is already included in
% your Linux system, but if not, obtain it from
% http://www.worldvisions.ca/wvdial/. 
% 
% You need to create the {\bf wvdial} startup file /etc/wvdial.conf.  You
% could do this simply by using the sample in the man page as a model.
% (Red Hat's {\bf wvdial} configuration tool is rp3; you may wish to run
% this instead.) Or you can use {\bf wvdial}'s own configuration tool; see
% Appendix \ref{wvdial} for details.


 
\end{document} 

