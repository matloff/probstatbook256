
\documentclass[11pt]{article}

\setlength{\oddsidemargin}{0in}
\setlength{\evensidemargin}{0in}
\setlength{\topmargin}{0.0in}
\setlength{\headheight}{0in}
\setlength{\headsep}{0in}
\setlength{\textwidth}{6.5in}
\setlength{\textheight}{9.0in}
\setlength{\parindent}{0in}
\setlength{\parskip}{0.1in}

\usepackage{times}
\usepackage{fancyvrb}  
\usepackage{relsize}  
\usepackage{hyperref}

\begin{document}

Let $I_i$ be 1 or 0, depending on whether the i$^{th}$ interaction is
successful or not, and set $S_n = I_1 +...+ I_n$ is the number of
successes among the first n interactions.  Then

\begin{equation}
E(S_n) = \sum_{i=1}^n \sigma_i
\end{equation}

and since the interactions are independent, 

\begin{equation}
Var(S_n) = \sum_{i=1}^n \sigma_i (1 - \sigma_i)
\end{equation}

Since $S_n$ is a sum of independent though nonidentically-distributed
random variables, the Central Limit Theorem (CLT) applies if certain
conditions, such as those of Lindeberg, are satisfied.  We do not pursue
those conditions here, but the CLT holds, one can find approximate
probabilities involving the $S_n$, as

\begin{equation}
P(S_n \leq k) \approx \Phi \left [(k-\sum_{i=1}^n \sigma_i)/
\sqrt{\sum_{i=1}^n \sigma_i (1 - \sigma_i)} \right ]
\end{equation}

\end{document}

