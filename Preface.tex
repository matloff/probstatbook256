\chapter*{Preface}
\label{preface}

Why is this book different from all other books on mathematical
probability and statistics?  The key aspect is the book's
consistently {\it applied} approach, especially important for
engineering students.

The applied nature comes is manifested in a number of senses.
First, there is a strong emphasis on intution, with less mathematical
formalism.  In my experience, defining probability via sample spaces,
the standard approach, is a major impediment to doing good applied work.
The same holds for defining expected value as a weighted average.
Instead, I use the intuitive, informal approach of long-run frequency
and long-run average.  I believe this is especially helpful when
explaining conditional probability and expectation, concepts that
students tend to have trouble with.  (They often think they understand
until they actually have to work a problem using the concepts.)

On the other hand, in spite of the relative lack of formalism, all
models and so on are described precisely in terms of random variables
and distributions.  And the material is actually somewhat more
mathematical than most at this level in the sense that it makes
extensive usage of linear algebra.

Second, the book stresses {\it real-world} applications.  Many similar
texts, notably the elegant and interesting book for computer science
students by Mitzenmacher, focus on probability, in fact discrete
probability.  Their intended class of ``applications'' is the
theoretical analysis of algorithms.  I instead focus on the actual use
of the material in the real world; which tends to be more continuous
than discrete, and more in the realm of statistics than probability.
This should prove especially valuable, as ``big data'' and machine
learning now play a significant role in applications of computers.

Third, there is a strong emphasis on modeling.  Considerable emphasis is
placed on questions such as: What do probabilistic models really mean,
in real-life terms?  How does one choose a model?  How do we assess the
practical usefulness of models?  This aspect is so important that there
is a separate chapter for this, titled Introduction to Model Building.
Throughout the text, there is considerable discussion of the real-world
meaning of probabilistic concepts.  For instance, when probability
density functions are introduced, there is an extended discussion
regarding the intuitive meaning of densities in light of the
inherently-discrete nature of real data, due to the finite precision of
measurement.  

Finally, the R statistical/data analysis language is used
throughout.  Again, several excellent texts on probability and
statistics have been written that feature R, but this book, by virtue of
having a computer science audience, uses R in a more sophisticated
manner.  My open source tutorial on R programming, {\it R for
Programmers} (\url{http://heather.cs.ucdavis.edu/~matloff/R/RProg.pdf}),
can be used as a supplement.  (More advanced R programming is covered in
my book, {\it The Art of R Programming}, No Starch Press, 2011.)

There is a large amount of material here.  For my one-quarter
undergraduate course, I usually cover Chapters 
\ref{probcalc},
\ref{dis},
\ref{dismarkov},
\ref{chap:contin},
\ref{chap:normal},
\ref{stopandreview},
\ref{randvec},
\ref{chap:statprologue},
\ref{chap:confints},
\ref{chap:sigtests},
\ref{chap:est} and
\ref{chap:linreg}.  
My lecture style is conversational, referring to
material in the book and making lots of supplementary remarks (``What if
we changed the assumption here to such-and-such?'' etc.).  Students read
the details on their own.  For my one-quarter graduate course, I cover
Chapters
\ref{stopandreview},
\ref{conmarkov},
\ref{chap:mix},
\ref{mar},
\ref{haz},
\ref{chap:nonpardens},
\ref{chap:mod},
\ref{chap:linreg},  
\ref{chap:class},  
\ref{chap:nonparregclass} and 
\ref{chap:among}.  

As prerequisites, the student must know calculus, basic matrix algebra, and
have some skill in programming.  As with any text in probability and
statistics, it is also necessary that the student has a good sense
of math intuition, and does not treat mathematics as simply memorization
of formulas.

The \LaTeX source {\bf .tex} files for this book are in
\url{http://heather.cs.ucdavis.edu/~matloff/132/PLN}, so readers can
copy the R code and experiment with it.  (It is not recommanded to
copy-and-paste from the PDF file, as hidden characters may be copied.)
The PDF file is searchable.

The following, among many, provided valuable feedback for which I am
very grateful:  Ibrahim Ahmed; Ahmed Ahmedin; Stuart Ambler; Earl Barr;
Benjamin Beasley; Matthew Butner; Michael Clifford; Dipak Ghosal; Noah
Gift; Laura Matloff; Nelson Max, Connie Nguyen, Jack Norman, Richard
Oehrle, Michael Rea, Sana Vaziri, Yingkang Xie, and Ivana Zetko.  The
cover picture, by the way, is inspired by an example in Romaine
Francois' old R Graphics Gallery, sadly now defunct.

Many of the data sets used in the book are from the UC Irvine Machine
Learning Repository, \url{http://archive.ics.uci.edu/ml/}.  Thanks to
UCI for making available this very valuable resource.

The book contains a number of references for further reading.  Since the
audience includes a number of students at my institution, the University
of California, Davis, I often refer to work by current or former UCD
faculty, so that students can see what their professors do in research.

This work is licensed under a Creative Commons Attribution-No Derivative
Works 3.0 United States License.  The details may be viewed at
\url{http://creativecommons.org/licenses/by-nd/3.0/us/}, but in essence
it states that you are free to use, copy and distribute the work, but
you must attribute the work to me and not ``alter, transform, or build
upon'' it.  If you are using the book, either in teaching a class or for
your own learning, I would appreciate your informing me.  I retain
copyright in all non-U.S. jurisdictions, but permission to use these
materials in teaching is still granted, provided the licensing
information here is displayed.  

