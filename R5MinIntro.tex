\chapter{R Quick Start}
\label{chap:rquickstart}

Here we present a five-minute introduction to the R data/statistical
programming language.  Further learning resources are available at
\url{http://heather.cs.ucdavis.edu/~/matloff/r.html}.

R syntax is similar to that of C. It is object-oriented (in the sense of
encapsulation, polymorphism and everything being an object) and is a
functional language (i.e. almost no side effects, every action is a
function call, etc.).

\section{Correspondences}

\begin{tabular}{|l|l|l|}
\hline
aspect & C & R \\ \hline 
\hline
assignment & = & \verb#<-# (or =) \\ \hline 
array terminology & array & vector (1-D), matrix (2-D), array (2-D+) \\ \hline 
array element & m[2,3] & m[2][3] \\ \hline 
storage & 2-D arrays in row-major order & matrices in column-major order \\ \hline 
mixed container &struct, members denoted by . & list, members denoted by \$ \\ \hline 
\end{tabular}

\section{Starting R}

To invoke R, just type ``R'' into a terminal window. On a Windows
machine, you probably have an R icon to click.  

If you prefer to run from an IDE, the easiest one for a quick install is
probably RStudio, \url{www.rstudio.org}.

R is (normally) interactive, with $>$ as the prompt.

\section{First Sample Programming Session}

Below is a commented R session, to introduce the concepts. I had a text
editor open in another window, constantly changing my code, then loading
it via R's {\bf source()} command.\footnote{I personally am not a big
fan of using IDEs for my programming activities.  If you use one, it
probably has a button to click as an alternative to using {\bf
source()}.} The original contents of the file {\bf odd.R} were:

\begin{lstlisting}[numbers=left]
oddcount <- function(x)  {
   k <- 0  # assign 0 to k
   for (n in x)  {
      if (n %% 2 == 1) k <- k+1  # %% is the modulo operator
   }
   return(k)
}
\end{lstlisting}

The R session is shown below.  You may wish to type it yourself as you
go along, trying little experiments of your own along the
way.\footnote{The source code for this file is at
\url{http://heather.cs.ucdavis.edu/~matloff/MiscPLN/R5MinIntro.tex}.}

\begin{lstlisting}[numbers=left]
> source("odd.R")  # load code from the given file
> ls()  # what objects do we have?
[1] "oddcount"
> # what kind of object is oddcount (well, we already know)?
> class(oddcount)
[1] "function"
> # can print any object by typing its name
> oddcount
function(x)  {
   k <- 0  # assign 0 to k
   for (n in x)  {
      if (n %% 2 == 1) k <- k+1  # %% is the modulo operator
   }
   return(k)
}
> # test it
> y <- c(5,12,13,8,88)  # c() is the concatenate function
> y
[1]  5 12 13  8 88
> oddcount(y)  # should report 2 odd numbers
[1] 2
> # change code to vectorize the count operation
> source("odd.R")
> oddcount
function(x)  {
   x1 <- (x %% 2) == 1  # x1 now a vector of TRUEs and FALSEs
   x2 <- x[x1]  # x2 now has the elements of x that were TRUE in x1
   return(length(x2))
}
> # try subset of y, elements 2 through 3
> oddcount(y[2:3])
[1] 1
> # try subset of y, elements 2, 4 and 5
> oddcount(y[c(2,4,5)])
[1] 0
> # compactify the code
> source("odd.R")
> oddcount
function(x)  {
   length(x[x %% 2 == 1])  # last value computed auto returned
}
> oddcount(y)
[1] 2
> # now have ftn return odd count AND the odd numbers themselves
> source("odd.R")
> oddcount
function(x)  {
   x1 <- x[x %% 2 == 1]
   return(list(odds=x1, numodds=length(x1)))
}
> # R's list type can contain any type; components delineated by $
> oddcount(y)
$odds
[1]  5 13

$numodds
[1] 2

> ocy <- oddcount(y)
> ocy
$odds
[1]  5 13

$numodds
[1] 2

> ocy$odds
[1]  5 13
\end{lstlisting}

Note that the R function {\bf function()} produces functions! Thus
assignment is used. For example, here is what {\bf odd.R} looked like at
the end of the above session:

\begin{lstlisting}[numbers=left]
oddcount <- function(x)  {
   x1 <- x[x %% 2 == 1]
   return(list(odds=x1, numodds=length(x1)))
}
\end{lstlisting}
  
We created some code, and then used function to create a function
object, which we assigned to {\bf oddcount}.

Note that we eventually {\bf vectorized} our function {\bf oddcount()}.
This means taking advantage of the vector-based, functional language
nature of R, exploiting R's built-in functions instead of loops.  This
changes the venue from interpreted R to C level, with a potentially
large increase in speed.  For example:

\begin{lstlisting}[numbers=left]
> x <- runif(1000000)  # 1000000 random numbers from the interval (0,1)
> system.time(sum(x))
   user  system elapsed 
  0.008   0.000   0.006 
> system.time({s <- 0; for (i in 1:1000000) s <- s + x[i]})
   user  system elapsed 
  2.776   0.004   2.859 
\end{lstlisting}

\section{Second Sample Programming Session}

A matrix is a special case of a vector, with added class attributes,
the numbers of rows and columns.

\begin{lstlisting}[numbers=left]
> # "rowbind() function combines rows of matrices; there's a cbind() too
> m1 <- rbind(1:2,c(5,8))
> m1
     [,1] [,2]
[1,]    1    2
[2,]    5    8
> rbind(m1,c(6,-1))
     [,1] [,2]
[1,]    1    2
[2,]    5    8
[3,]    6   -1
> m2 <- matrix(1:6,nrow=2)
> m2
     [,1] [,2] [,3]
[1,]    1    3    5
[2,]    2    4    6
> ncol(m2)
[1] 3
> nrow(m2)
[1] 2
> m2[2,3]
[1] 6
# get submatrix of m2, cols 2 and 3, any row
> m3 <- m2[,2:3]
> m3
     [,1] [,2]
[1,]    3    5
[2,]    4    6
> m1 * m3  # elementwise multiplication
     [,1] [,2]
[1,]    3   10
[2,]   20   48
> 2.5 * m3  # scalar multiplication (but see below)
     [,1] [,2]
[1,]  7.5 12.5
[2,] 10.0 15.0
> m1 %*% m3  # linear algebra matrix multiplication
     [,1] [,2]
[1,]   11   17
[2,]   47   73
> # matrices are special cases of vectors, so can treat them as vectors 
> sum(m1)
[1] 16
> ifelse(m2 %%3 == 1,0,m2) # (see below)
     [,1] [,2] [,3]
[1,]    0    3    5
[2,]    2    0    6
\end{lstlisting}

The ``scalar multiplication'' above is not quite what you may think,
even though the result may be.  Here's why:

In R, scalars don't really exist; they are just one-element vectors.
However, R usually uses {\bf recycling}, i.e. replication, to make
vector sizes match.  In the example above in which we evaluated the
express \verb^2.5 * m3^, the number 2.5 was recycled to the matrix 

\begin{equation}
\left (
\begin{array}{rr}
2.5 & 2.5 \\
2.5 & 2.5 
\end{array}
\right )
\end{equation}

in order to conform with {\bf m3} for (elementwise) multiplication.

The {\bf ifelse()} function's call has the form

\begin{lstlisting}
ifelse(boolean vectorexpression1, vectorexpression2, vectorexpression3)
\end{lstlisting}

All three vector expressions must be the same length, though R will
lengthen some via recycling.  The action will be to return a vector of
the same length (and if matrices are involved, then the result also has
the same shape).  Each element of the result will be set to its
corresponding element in {\bf vectorexpression2} or {\bf
vectorexpression3}, depending on whether the corresponding element in
{\bf vectorexpression1} is TRUE or FALSE.

In our example above,

\begin{lstlisting}
> ifelse(m2 %%3 == 1,0,m2) # (see below)
\end{lstlisting}

the expression \verb^m2 %%3 == 1^ evaluated to the boolean matrix

\begin{equation}
\left (
\begin{array}{rrr}
T & F & F\\
F & T & F
\end{array}
\right )
\end{equation}

(TRUE and FALSE may be abbreviated to T and F.)

The 0 was recycled to the matrix

\begin{equation}
\left (
\begin{array}{rrr}
0 & 0 & 0 \\
0 & 0 & 0
\end{array}
\right )
\end{equation}

while {\bf vectorexpression3}, {\bf m2}, evaluated to itself.

\section{Online Help} 

R's {\bf help()} function, which can be invoked also with a
question mark, gives short descriptions of the R functions. For
example, typing

\begin{Verbatim}[fontsize=\relsize{-2}]
> ?rep
\end{Verbatim}
  
will give you a description of R's {\bf rep()} function.

An especially nice feature of R is its {\bf example()} function, which
gives nice examples of whatever function you wish to query. For
instance, typing

\begin{lstlisting}
> example(wireframe())
\end{lstlisting}
  
will show examples---R code and resulting pictures---of {\bf wireframe()},
one of R's 3-dimensional graphics functions.
