\documentclass[twocolumn]{article}

\setlength{\oddsidemargin}{-0.5in}
\setlength{\evensidemargin}{-0.5in}
\setlength{\topmargin}{0.0in}
\setlength{\headheight}{0in}
\setlength{\headsep}{0in}
\setlength{\textwidth}{7.0in}
\setlength{\textheight}{9.5in}
\setlength{\parindent}{0in}
\setlength{\parskip}{0.05in}
\setlength{\columnseprule}{0.3pt}
\usepackage{fancyvrb}
\usepackage{relsize}
\usepackage{hyperref}

\begin{document}

Name: \_\_\_\_\_\_\_\_\_\_\_\_\_\_\_\_\_\_\_\_\_\_\_\_\_\_\_\_

Directions: {\bf Work only on this sheet} (on both sides, if needed); do not
turn in any supplementary sheets of paper. There is actually plenty of room
for your answers, as long as you organize yourself BEFORE starting writing.

{\bf 1.} (20) The statement

\begin{Verbatim}[fontsize=\relsize{-2}]
 __asm__("addl $8,%eax");
\end{Verbatim}

would be mostly likely to appear amongst the code of what language?

{\bf 2.} (20) Fill in the blank:  The term
\_\_\_\_\_\_\_\_\_\_\_\_\_\_\_\_\_\_\_\_\_\_\_\_\_\_ refers to a set of
parallel wires, to which the CPU and memory are attached.

{\bf 3.} (20) Fill in the blanks:  In the instruction on line 50, p.73,
the source and destination operands are specified in
\_\_\_\_\_\_\_\_\_\_\_\_\_\_\_\_\_\_\_\_\_\_\_\_\_\_ and
\_\_\_\_\_\_\_\_\_\_\_\_\_\_\_\_\_\_\_\_\_\_\_\_\_\_ modes,
respectively.

{\bf 4.} I had a certain assembly language source file {\bf trymovs.s},
which I assembled, linked and fed into GDB.  I set one breakpoint, and
ran the program.  At the breakpoint, I queried the contents of the
registers.  Here is what my GDB session looked like at that point:

\begin{Verbatim}[fontsize=\relsize{-2}]
(gdb) break 14
Breakpoint 1 at 0x8048092: file trymovs.s, line 14.
(gdb) r
Starting program: ...
Breakpoint 1, _start () at trymovs.s:14
14	    rep movsb          
Current language:  auto; currently asm
(gdb) info registers
eax            0x8888	34952
ecx            0x888888	8947848
edx            0x8888888	143165576
ebx            0x88888	559240
esp            0xbfc3e230	0xbfc3e230
ebp            0x0	0x0
esi            0x8049094	134516884
edi            0x804909a	134516890
eip            0x8048092	0x8048092 <_start+30>
eflags         0x292	[ AF SF IF ]
cs             0x73	115
ss             0x7b	123
ds             0x7b	123
es             0x7b	123
fs             0x0	0
gs             0x0	0
\end{Verbatim}

Note that the second and third columns show register contents, in hex
and decimal.

\begin{itemize}

\item [(a)] (15) How many bytes will be copied?

\item [(b)] (15) Now resuming execution, what is the first value to go
onto the address bus?  (This is {\it execution}, not counting
instruction fetch.)

\end{itemize}

{\bf 5.} (10) The R language allows one to interface with C code in the
following way.  One writes a C function, say {\bf cftn()} in C, then
compiles it into a library.  Then, while running R, one calls the R
function {\bf dyn.load()}, with the argument being the library file.
One can then call {\bf cftn()} from R.  On Linux, what suffix will that
file name have?

{\bf Solutions:}

{\bf 1.}  C++ (Sec. 3.14)

{\bf 2.} {\it bus} (p.44)

{\bf 3.} immediate, register (Sec. 3.7)

{\bf 4a.} c(ECX) = 8947848 (p.99)

{\bf 4b.} c(EDI) = 134516890 (p.99, p.44)

{\bf 5.} {\bf .so} (p.95)

\end{document}

