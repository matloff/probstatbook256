\documentclass[twocolumn]{article}

\setlength{\oddsidemargin}{-0.5in}
\setlength{\evensidemargin}{-0.5in}
\setlength{\topmargin}{0.0in}
\setlength{\headheight}{0in}
\setlength{\headsep}{0in}
\setlength{\textwidth}{7.0in}
\setlength{\textheight}{9.5in}
\setlength{\parindent}{0in}
\setlength{\parskip}{0.05in}
\setlength{\columnseprule}{0.3pt}
\usepackage{fancyvrb}
\usepackage{relsize}
\usepackage{hyperref}
\usepackage{listings}

\usepackage{amsmath}


\begin{document}

Name: \_\_\_\_\_\_\_\_\_\_\_\_\_\_\_\_\_\_\_\_\_\_\_\_\_\_\_\_

Directions: {\bf \Large Work only on this sheet} (on both sides, if
needed); do not turn in any supplementary sheets of paper. There is
actually plenty of room for your answers, as long as you organize
yourself BEFORE starting writing.

{\bf 1.} (20) Fill in the blank with a term from our course:  The value on
the left-hand side of (6.1) turns out not to depend on t, in the case of
W being exponentially distributed.  We say that the W has a constant
\ \ \ \ \ \ \ \ \ \ \ \ \ \ \ \ \ \  function.

{\bf  2.}  Suppose $W_1$, $W_2$ and $W_3$ are independent, each with
distribution U(0,1).

\begin{itemize}

\item [(a)] (20) Write (but do not evaluate) an integral for $P(W_1+W_2 <
0.8)$.

\end{itemize}

In parts (b) and (c), suppose we're interested in finding
$Cov(W_1+W_2,W_1+W_3)$, using (5.107).

\begin{itemize}

\item [(b)] (20) Show the matrix A.

\item [(c)] (20) Show the matrix Cov(W).

\end{itemize}

{\bf 3.} (20) Suppose Pei and Gowtham each take random samples of size 2 with
replacement from the three-person population in the toy example on
p.185.  Find the probability that Gowtham's sample mean is exactly equal
to Pei's.  {\bf EXPRESS YOUR ANSWER AS A SINGLE FRACTION, REDUCED TO
LOWEST TERMS}, but show your work!

\onecolumn

{\bf Solutions:}

{\bf 1.} hazard

{\bf 2.a}

$$
\int_{0}^{0.8} 
\int_{0}^{0.8-s} 
1 \cdot 1 ~ dt ~ ds
$$

{\bf 2.b}

$$
\left (
\begin{array}{rrr}
    1 & 1 & 0 \\
    1 & 0 & 1 \\
\end{array}
\right )
$$

{\bf 2.c}  A U(0,1) distribution has variance 1/12 and the $W_i$ are
independent.  So the covariance matrix is diagonal, with all diagonal
elements equal to 1/12.

{\bf 3.} 
$$
(\frac{1}{9})^2 +
(\frac{2}{9})^2 +
(\frac{2}{9})^2 +
(\frac{1}{9})^2 +
(\frac{2}{9})^2 +
(\frac{1}{9})^2 
= 5/27
$$

\end{document}

