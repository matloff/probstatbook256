\documentclass{article}

\setlength{\oddsidemargin}{-0.5in}
\setlength{\evensidemargin}{-0.5in}
\setlength{\topmargin}{0.0in}
\setlength{\headheight}{0in}
\setlength{\headsep}{0in}
\setlength{\textwidth}{7.0in}
\setlength{\textheight}{9.5in}
\setlength{\parindent}{0in}
\setlength{\parskip}{0.05in}
\setlength{\columnseprule}{0.3pt}
\usepackage{fancyvrb}
\usepackage{relsize}
\usepackage{hyperref}

\begin{document}

Name: \_\_\_\_\_\_\_\_\_\_\_\_\_\_\_\_\_\_\_\_\_\_\_\_\_\_\_\_

Directions: {\bf \Large Work only on this sheet} (on both sides, if
needed); do not turn in any supplementary sheets of paper. There is
actually plenty of room for your answers, as long as you organize
yourself BEFORE starting writing.

% {\bf \Large Unless otherwise stated, give numerical answers as
% expressions, e.g. $\frac{2}{3} \times 6 - 1.8$.  Do NOT use
% calculators.}

{\bf 1.} () Consider some probabilistic model in which there are events
A and B.  For each of the following, write T if the equation or
inequality is always true, and write F otherwise:

\begin{itemize}

\item [(a)] $P(A) = P(B)$

\item [(b)] $P(A) \geq P(A \textrm{ and } B)$

\item [(c)] $P(A \textrm{ and }  B) = P(A | B)$

\item [(d)] $P(A \textrm{ and } B) \leq P(A \textrm{ or } B)$

\end{itemize}

{\bf 2.} () In the ALOHA example, consider each of the situations below.
Assume that p and q are both strictly between 0 and 1, exclusive.  For
each situation, write P if it is possible, I if it is impossible:

\begin{itemize}

\item [(a)] $X_1 = X_2 = 2$

\item [(b)] $X_1 = 0$

\item [(c)] $X_2 = 0$

\item [(d)] $X_2 = 1$

\end{itemize}

{\bf Solutions:}

{\bf 1.a} F, of course

{\bf 1.b} T; within a row of the notebook, if the ``A and B'' column
says Yes, then the ``A'' column must say yes too, so the latter column's
long-run proportion of Yes entries will be at least that of the former
column

{\bf 1.c} F, of course; emphasized in class

{\bf 1.d} T; same reasoning as for (b) above

{\bf 2.a} P; the two terminal could both back off in both epochs

{\bf 2.b} I; at most one message can be sent per epoch, so $X \geq 1$

{\bf 2.c} P; could have one message sent in the first epoch, then no new
messages created in the second, followed by a successful transmission of
the remaining message in the second epoch

{\bf 2.d} P; see (c) above

\end{document}

