\chapter{Classification} 
\label{chap:class}

% \section{Nonlinear Parametric Regression Models}
% \label{nonlin}
% 
% We pointed out in Section \ref{whatmeans} that the word {\it linear} in
% {\it linear regression model} means linear in $\beta$, not in t.  This
% is the most popular approach, as it is computationally easy, but
% nonlinear models are often used.
% 
% The most famous of these is the {\bf logistic} model, for the case in
% which $Y$ takes on only the values 0 and 1.  As we have seen before
% (Section \ref{indicator}), in this case the expected value becomes a
% probability.  The logistic model for a nonvector $X$ is then
% 
% \begin{equation}
% \label{logit}
% m_{Y;X}(t) = P(Y = 1 | X = t) = \frac{1}{1+e^{-(\beta_0+\beta_1 t})}
% \end{equation}
% 
% It extends to the case of vector-valued $X$ in the obvious way. 
% 
% The logistic model is quite widely used in computer science, in medicine,
% economics, psychology and so on.  We'll return to this model in Section
% \ref{logitsec}.
% 
% Here is an example of a nonlinear model used in kinetics of chemical
% reactions, with r = 3: \footnote{See
% \url{http://www.mathworks.com/access/helpdesk/help/toolbox/stats/rsmdemo.html}.}
% 
% \begin{equation}
% m_{Y;X}(t) = \frac
% {\beta_1 t^{(2)} - t^{(3)}/\beta_5}
% {1+\beta_2 t^{(1)} + \beta_3 t^{(2)} + \beta_4 t^{(3)}}
% \end{equation}
% 
% Here the X vector is (hydrogen, n-pentane, isopentane) and Y is the
% reaction rate.
% 
% Unfortunately, in most cases, the least-squares estimates of the
% parameters in nonlinear regression do not have closed-form solutions,
% and numerical methods must be used.  But R does that for you, via the
% {\bf nls()} function in general, and via {\bf glm()} for the logistic
% and related models in particular.

In prediction problems, in the special case in which Y is an indicator
variable, with the value 1 if the object is in a class and 0 if not, the
regression problem is called the {\bf classification
problem}.\footnote{The case of c $>$ 2 classes will be treated in
Section \ref{multiclass}.}

We'll formalize this idea in Section \ref{meanisprob}, but first,
here are some examples:

\begin{itemize}

\item A forest fire is now in progress.  Will the fire reach a certain
populated neighborhood?  Here Y would be 1 if the fire reaches the
neighborhood, 0 otherwise.  The predictors might be wind direction,
distance of the fire from the neighborhood, air temperature and
humidity, and so on.

\item Is a patient likely to develop diabetes?  This problem has been
studied by many researchers, e.g.  Using Neural Networks To Predict the
Onset of Diabetes Mellitus, Murali S. Shanker {\it J. Chem. Inf. Comput.
Sci.}, 1996, 36 (1), pp 35–41.  A famous data set involves Pima Indian
women, with Y being 1 or 0, depending on whether the patient does
ultimately develop diabetes, and the predictors being the number of
times pregnant, plasma glucose concentration, diastolic blood pressure,
triceps skin fold thickness, serum insulin level, body mass index,
diabetes pedigree function and age. 

\item Is a disk drive likely to fail soon?  This has been studied for
example in Machine Learning Methods for Predicting Failures in Hard
Drives: A Multiple-Instance Application, by Joseph F. Murray, Gordon F.
Hughes, and Kenneth Kreutz-Delgado, {\it Journal of Machine Learning
Research} 6 (2005) 783-816.  Y was 1 or 0, depending on whether the
drive failed, and the predictors were temperature, number of read
errors, and so on.

\item An online service has many customers come and go.  It would like
to predict who is about to leave, so as to offer them a special deal for
staying with this firm.

\item Of course, a big application is character recognition, based on
pixel data.  This is different from the above examples, as there are
more than two classes, many more.  We'll return to this point soon.

\end{itemize}

In electrical engineering the classification is called {\bf pattern
recognition}, and the predictors are called {\bf features}.  In computer
science the term {\bf machine learning} usually refers to classification
problems.  Different terms, same concept.

\section{Classification = Regression}
\label{meanisprob}

All of the many machine learning algorithms, despite their
complexity, really boil down to regression at their core.  Here's why:

\subsection{What Happens with Regression in the Case Y = 0,1?}

As we have frequently noted the mean of any indicator random variable is
the probability that the variable is equal to 1 (Section
\ref{indicator}).  Thus in the case in which our response variable Y
takes on only the values 0 and 1, i.e.  classification problems, the
regression function reduces to

\begin{equation}
m_{Y;X}(t) = P(Y = 1 | X = t)
\end{equation}

(Remember that X and t are vector-valued.)

As a simple but handy example, suppose Y is gender (1 for male, 0 for
female), $X^{(1)}$ is height and $X^{(2)}$ is weight, i.e. we are
predicting a person's gender from the person's height and weight.  Then
for example, $m_{Y;X}(70,150)$ is the probability that a person of
height 70 inches and weight 150 pounds is a man.  Note again that this
probability is a population fraction, the fraction of men among all
people of height 70 and weight 150 in our population.

Make a mental note of the optimal prediction rule, if we know the
population regression function:

\begin{quote}
Given X = t, the optimal prediction rule is to predict that Y = 1 if and
only if $m_{Y;X}(t) > 0.5$.
\end{quote}

So, if we known a certain person is of height 70 and weight 150, our
best guess for the person's gender is to predict the person is male
if and only if $m_{Y;X}(70,150) > 0.5$.\footnote{Things change in the
multiclass case, though, as will be seen in Section \ref{multiclass}.}

The optimality makes intuitive sense, and is shown in Section
\ref{optimal2}.   

\section{Logistic Regression:  a Common Parametric Model for the
Regression Function in Classification Problems}
\label{logitsec}

Remember, we often try a parametric model for our regression function
first, as it means we are estimating a finite number of quantities,
instead of an infinite number.  Probably the most commonly-used model is
that of the {\bf logistic function} (often called ``logit'').  Its
r-predictor form is 

\begin{equation}
\label{logit2}
m_{Y;X}(t) = P(Y = 1 | X = t) = \frac{1}{1+e^{-(\beta_0+\beta_1
t_1+...+\beta_r t_r)}}
\end{equation}

For instance, consider the patent example in Section \ref{examples}.
Under the logistic model, the population proportion of all patents that
are publicly funded, among those that contain the word ``NSF,'' do not
contain ``NIH,'' and make five claims would have the value

\begin{equation}
\frac{1}{1+e^{-(\beta_0 + \beta_1 + 5\beta_3)}}
\end{equation}

\subsection{The Logistic Model:  Motivations}
\label{logitmotivations}

The logistic function itself, 

\begin{equation}
\frac{1}{1+e^{-u}}
\end{equation}

has values between 0 and 1, and is thus a good candidate for modeling a
probability.  Also, it is monotonic in u, making it further attractive,
as in many classification problems we believe that $m_{Y;X}(t)$ should
be monotonic in the predictor variables.

But there are additional reasons to use the logit model, as it includes
many common parametric models for X.  To see this, note that we can
write, for vector-valued discrete X and t,

\begin{eqnarray}
\label{oneonelogodds}
P(Y = 1 | X = t) &=&  \frac{P(Y = 1 ~ \textrm{and} ~ X = t)}{P(X = t)} \\
&=& \frac{P(Y = 1) P(X = t | Y = 1)}{P(X = t)} \\
&=& \frac{P(Y = 1) P(X = t | Y = 1)}
{P(Y = 1) P(X = t | Y = 1) + P(Y = 0) P(X = t | Y = 0)} \\
\label{intermedone}
&=& \frac{1}
{1 + \frac{(1-q) P(X = t | Y = 0)}{q P(X = t | Y = 1)}}
\label{finaloneone}
\end{eqnarray}

where $q = P(Y = 1)$ is the proportion of members of the population which
have $Y = 1$.  (Keep in mind that this probability is unconditional!!!!
In the patent example, for instance, if say $q = 0.12$, then 12\% of all
patents in the patent population---without regard to words used, numbers
of claims, etc.---are publicly funded.) 

% \checkpoint

If $X$ is a continuous random vector, then the analog of
(\ref{finaloneone}) is

\begin{equation}
\label{oneonelogodds2}
P(Y = 1 | X = t) = \frac{1}
{1 + \frac{(1-q) f_{X|Y=0}(t)}{q f_{X|Y=1}(t)}}
\end{equation}

Now for simplicity, suppose $X$ is scalar, i.e. r = 1.  And suppose
that, given Y, X has a normal distribution.  In other words, within each
class, $Y$ is normally distributed.  Suppose also that the two
within-class variances of X are equal, with common value $\sigma^2$, but
with means $\mu_0$ and $\mu_1$.  Then

\begin{equation}
f_{X|Y=i}(t) = \frac{1}{\sqrt{2\pi\sigma}}
\exp{\left [ -0.5 \left (\frac{t-\mu_i}{\sigma} \right )^2 \right ]}
\end{equation}

After doing some elementary but rather tedious algebra,
(\ref{oneonelogodds2}) reduces to the logistic form

\begin{equation}
\label{logitform}
\frac{1}{1+e^{-(\beta_0 + \beta_1 t)}}
\end{equation}

where 

\begin{equation}
\label{logitbeta0}
\beta_0 = -\ln{\left ( \frac{1-q}{q} \right )} + 
\frac{\mu^2_0 - \mu^2_1}{2\sigma^2},
\end{equation}

and

\begin{equation}
\label{logitbeta1}
\beta_1 = \frac{\mu_1 - \mu_0}{\sigma^2},
\end{equation} 

{\bf In other words, if X is normally distributed in both classes, with
the same variance but different means, then $m_{Y;X}()$ has the logistic
form!} And the same is true if X is multivariate normal in each class,
with different mean vectors but equal covariance matrices.  (The algebra
is even more tedious here, but it does work out.)  Given the central
importance of the multivariate normal family---the word {\it central}
here is a pun, alluding to the (multivariate) Central Limit
Theorem---this makes the logit model even more useful.

If you retrace the derivation above, you will see that the logit model
will hold for any within-class distributions for which 

\begin{equation}
\label{logitlogodds}
\ln \left ( \frac{f_{X|Y=0}(t)}{f_{X|Y=1}(t)} \right )
\end{equation}

(or its discrete analog) is linear in t.  Well guess what---this
condition is true for exponential distributions too!  Work it out for
yourself.

% \checkpoint

In fact, a number of famous distributions imply the logit model.  So,
logit is not only a good intuitive model, as discussed above, but in
addition there are some good theoretical recommendations for it.

\subsection{Esimation and Inference for Logit Coefficients}

We fit a logit model in R using the {\bf glm()} function, with the
argument {\bf family=binomial}.  The function finds Maximum Likelihood
Estimates (Section \ref{mle}) of the $\beta_i$.\footnote{As in the case
of linear regression, estimation and inference are done conditionally on
the values of the predictor variables $X_i$.} 

The output gives standard errors for the $\widehat{\beta}_i$ as in the
linear model case.  This enables the formation of confidence intervals
and significance tests on individual $\widehat{\beta}_i$.  For inference
on linear combinations of the $\widehat{\beta}_i$, use the {\bf vcov()}
function as in the linear model case.

\section{Example:  Forest Cover Data}
\label{forest}

Let's look again at the forest cover data we saw in Section
\ref{forestcover}.\footnote{There is a problem here, to be discussed in
Section \ref{ymarg}, but which will not affect the contents of this
section.} Recall that this application has the Prediction goal, rather
than the Description goal;\footnote{Recall these concepts from Section
\ref{goals}.} we wish to predict the type of forest cover.  There were
seven classes of forest cover.  

\subsubsection{R Code}

For simplicity, I restricted my analysis to classes 1 and
2.\footnote{This will be generalized in Section \ref{multiclass}.}  In
my R analysis I had the class 1 and 2 data in objects {\bf cov1} and
{\bf cov2}, respectively.  I combined them,

\begin{Verbatim}[fontsize=\relsize{-2}]
> cov1and2 <- rbind(cov1,cov2)
\end{Verbatim}

and created a new variable to serve as Y, recoding the 1,2 class names
to 1,0:

\begin{Verbatim}[fontsize=\relsize{-2}]
cov1and2[,56] <- ifelse(cov1and2[,55] == 1,1,0)
\end{Verbatim}

Let's see how well we can predict a site's class from the variable HS12
(hillside shade at noon) that we investigated in Chapter
\ref{chap:sigtests}, using a logistic model.

As noted earlier, in R we fit logistic models via the {\bf glm()}
function, for generalized linear models.  The word {\it generalized}
here refers to models in which some function of $m_{Y;X}(t)$ is linear
in parameters $\beta_i$.  For the classification model, 

\begin{equation}
\ln \left (  m_{Y;X}(t) / [1-m_{Y;X}(t)] \right )
= \beta_0 + \beta_1 t^{(1)} + ... + \beta_r t^{(r)}
\end{equation}

(Recall the discussion surrounding (\ref{logitlogodds}).)

This kind of generalized linear model is specified in R by setting the
named argument {\bf family} to {\bf binomial}.  Here is the call:

\begin{Verbatim}[fontsize=\relsize{-2}]
> g <- glm(cov1and2[,56] ~ cov1and2[,8],family=binomial)
\end{Verbatim}

The result was:

\begin{Verbatim}[fontsize=\relsize{-2}]
> summary(g)

Call:
glm(formula = cov1and2[, 56] ~ cov1and2[, 8], family = binomial)

Deviance Residuals:
   Min      1Q  Median      3Q     Max
-1.165  -0.820  -0.775   1.504   1.741

Coefficients:
               Estimate Std. Error z value Pr(>|z|)
(Intercept)    1.515820   1.148665   1.320   0.1870
cov1and2[, 8] -0.010960   0.005103  -2.148   0.0317 *
---
Signif. codes:  0 ‘***’ 0.001 ‘**’ 0.01 ‘*’ 0.05 ‘.’ 0.1 ‘ ’ 1

(Dispersion parameter for binomial family taken to be 1)

    Null deviance: 959.72  on 810  degrees of freedom
Residual deviance: 955.14  on 809  degrees of freedom
AIC: 959.14

Number of Fisher Scoring iterations: 4
\end{Verbatim}

\subsection{Analysis of the Results}

You'll immediately notice the similarity to the output of {\bf lm()}.
In particular, note the Coefficients section.  There we have the
estimates of the population coefficients $\beta_i$, their standard
errors, and p-values for the tests of $H_0: \beta_i = 0$.

One difference from the linear case is that in that case, the tests of 

\begin{equation}
 H_0:  \beta_i = 0
\end{equation}

were ``exact,'' based on the Student-t distribution, rather than being
approximate tests based on the Central Limit Theorem.  The assumption is
that the conditional distribution of the response given the predictors
is exactly normal.  As noted before, those tests can't possibly be
exact, since the assumption cannot exactly hold.

But in the logit case, no ``exact'' test is available anyway, so R does
indeed do approximate tests based on the Central Limit Theorem.
Accordingly, the test column in the output is labeled ``z value,''
rather than ``t value'' as before.

At any rate, we see that for example $\widehat{\beta}_1 = -0.01$.  This
is tiny, reflecting our analysis of this data in Chapter
\ref{chap:sigtests}.  There we found that the estimated mean values of
HS12 for cover types 1 and 2 were 223.8 and 226.3, a difference of only
2.5, minuscule in comparison to the estimated means themselves.  That
difference in essence now gets multiplied by 0.01.  Let's see the effect
on the regression function, i.e. the probability of cover type 1 given
HS12.

Note first, though, that things are a little more complicated than they
were in the linear case.  Recall our first baseball data analysis, in
Section \ref{baseball1}.  Our model for $m_{Y;X}()$ was

\begin{equation}
\textrm{mean weight = } 
\beta_0 + \beta_1 \textrm{ height}
\end{equation}

After estimating the $\beta_i$ from the data, we had the estimated
regression equation,

\begin{lstlisting}
mean weight = -155.092 + 4.841 height
\end{lstlisting}

In presenting these results to others, we can illustrate the above
equation by noting, for instance, that a 3-inch difference in height
corresponds to an estimated $3 \times 4.841 = 14.523$ difference in mean
weight.  The key point, though, is that this difference is the same
whether we are comparing 70-inch-tall players with 73-inch-tall ones, or
comparing 72-inch-tall players with 75-inch-tall ones, etc.  

In our nonlinear case, the logit, the different in regression value will
NOT depend only on the difference between the predictor values, in this
case HS12.  We cannot simply say something like ``If two forest sites
differ in HS12 by 15, then the difference in probability of cover type 1
differs by such-and-such an amount.''  

Instead we'll have to choose two specific HS11 values for our
illustration.  Let's use the estimated mean HS12 values for the two
cover types, 223.8 and then 226.3, found earlier.  

So, in (\ref{logit2}), let's plug in our estimates 1.52 and -0.01 from
our R output above, twice, first with HS12 = 223.8 and then with HS12 =
226.3   

In other words, let's imagine two forest sites, with unknown cover type,
but known HS12 values 223.8 and 226.8 that are right in the center of
the HS12 distribution for the two cover types.  What would we predict
for the cover types to be for those two sites?

Plugging in to (\ref{logit2}), the results are 0.328 and 0.322,
respectively.  Remember, these numbers are the estimated probabilities
that we have cover type 1, given HS12.  So, our guess---predicting
whether we have cover type 1 or 2---isn't being helped much by knowing
HS12.

In other words, HS12 isn't having much effect on the probability of
cover type 1, and so it cannot be a good predictor of cover type.  

{\bf And yet...} the R output says that $\beta_1$ is ``significantly''
different from 0, with a p-value of 0.03.  Thus, we see once again that
significance testing does not achieve our goal.  

\section{The Multiclass Case} 
\label{multiclass}

In classification problems, we often have more than two classes.  In the
forest cover example above, we simplified to having just two types of
forest cover, but there were actually seven.  In an optical character
recognition application, there may be dozens of classes, maybe even
hundreds or thousands for some written languages.

We will use $m$ to denote the number of classes, and code the classes as
$0,1,...,m-1$.  $Y$ will the class number, and let

\begin{equation}
q_i = P(Y = i) = P(\textrm{class i}), i = 0,1,...,m-1
\end{equation}

Note that the $q_i$ are {\it un}conditional probabilities.

\subsection{One vs. All Approach}

The methods remain basically the same.  Consider the forest cover
example, for instance.  There are seven cover types, so we could run
seven logistic models, predicting a dummy Y for each type.  In the first
run, Y would be 1 for cover type 1, 0 for everything else.  In
predicting a new case, we plug the predictor values into each of the
seven models, and guess Y to be the one with maximal probability among
the seven.

The above approach is called One versus All (OVA).  To see where the name
comes from, again consider the forest cover example, with seven logistic
regression models.  In the i$^{th}$ model, we are predicting class i,
against the collective alternative of all the other classes.  

There is also the All versus All method (AVA). Here we run a regression
analysis for each possible pair of classes ($7 \times 6 / 2 = 21$ of
them for the forest cover data).  For exach pair, we restrict our
analysis only to data for the two classes.  In predicting a new case, we
use a ``voting'' procedure:  For each pair of classes, we decide which
of the two classes is more likely for this new case, and then take our
guess for the class to be whichever class ``won'' the most times.

AVA has the disadvantage of using much less data for each pair. But if
for example we are using a logistic model, which implies a linear
boundary between each class and all the others,\footnote{The boundary is
implied by setting (\ref{logitform}) to 0.5.}, an assumption that may
not be very accurate. AVA essentially makes the weaker assumption that
the boundary is linear between each {\it pair} of classes.

{\it Note that under either OVA or AVA, we are always predicting one
class at a time.}  Let's call this the Two Classes Suffice Principle,
which will have implications below.

There are multivariate versions of the logit model, but they make
certain restrictive assumptions.

\subsection{Issues of Data Balance}
\label{ymarg}

Here will discuss a topic that is usually  glossed over in treatments of
the classification problem, and indeed is often misunderstood in the
machine learning (ML) research literature, where one sees much
questionable handwringing over ``the problem of unbalanced data.''  On
the contrary, the real problem is often that the data are {\it
balanced}.

\subsubsection{Statement of the Problem}

Consider the UCI Letters Recognition data set
(\url{https://archive.ics.uci.edu/ml/datasets/Letter+Recognition}).
There are between 700 and 800 cases for each English capital letter,
which does not reflect that wide variation in letter frequencies.  The
letter 'E', for example, is more than 100 times as frequent as the
letter 'Z', according to published data (see below).  This would be
considered ``balanced'' data, but it would be artificially so.

Now think of a different version of that data, in which the
letter frequencies reflect their true nature in general English.  There
would be rather few instances of 'Z' here, but lots of cases of 'A'.
This would be ``unbalanced'' but natural.  And our predictions for data
as a whole would be optimal in the sense of minimizing overall
misclassification error (Section \ref{optimal2}).  

From here on, we will use the term {\it unbalanced} to mean that our
data form a sample from our target population, in this case the
population of all English text. Then we can estimate the $q_i$ from the
data:

\begin{equation}
\label{qihat}
\widehat{q}_i = \frac{N_i}{n}
\end{equation}

where $N_i$ is the number of our data points in class $i$, and $n$ is
the total number of observations in the data set.  For balanced data,
our data arose from sampling independently from the various classes,
without attempting to have our data's class frequencies be
representative of the population.

In the balanced case, due to a paucity of data for 'Z', our
misclassification rate for that particular letter (guessing another
letter when the letter is actually 'Z') might not be good.  The latter
issue is the one of concern in the machine learning community.

In other words, for general prediction accuracy, unbalanced data is the
best.  If we are concerned with misclassification rates for specific
classes, things are more complex.

\subsubsection{Solutions}

There are really two problems to be solved:

\begin{itemize}

\item [(a)] Our data is balanced, when it shouldn't be, as with the Letter
Recognition data.

\item [(b)] Our data is unbalanced, and we are concerned about
misclassification with the rarer classes.

\end{itemize}

We will assume just two classes here, per the Two Classes Suffice
Principle we set up earlier.  Let $q$ and $1-q$ denote the class
probabilities for $Y = 1$ and $Y = 0$.

Let's first treat the context of Problem (a), and consider the quantity
$q$ in (\ref{oneonelogodds2}). If our data were balanced, we actually
has an easy solution, if we know the correct value of $q$ from some
external source. For instance, in the English letters example above,
there is much published data, such as at the Web page Practical
Cryptography (\url{http://practicalcryptography.com/cryptanalysis/
letter-frequencies-various-languages/english-letter-frequencies/}). They
find that $\pi_A = 0.08$, $\pi_B = 0.0160$, $\pi_C = 0.0316$ and so on.
So, if we use logistic regression (26 times), then when we run the one
for 'C', we take $q = 0.0316$.

Now recall from Section \ref{logitmotivations} that one of the
motivations for the logistic model is that it holds if the density of
our predictor vector $X$ is multivariate normal within each class (with
equal covariance matrices in each class).  It turns out that Equation
(\ref{logitbeta0}) holds for a general number of predictors, not just 1
as in that equation.  

If we then run {\bf glm()} for our logistic regression, that function
will act as if our data is unbalanced (the situation it was designed
for).  That means that our $\widehat{\beta}_0$ term will be incorrect,
essentially using (\ref{qihat}) .  This suggests the following
adjustment procedure.

\begin{itemize}

\item[(i)] Add $\ln(Ni0/n)$.

\item[(ii)] Subtract $\ln[(1-\pi)/\pi)]$, where $\pi$ is the true class
probability.

\end{itemize}

If we use a nonparametric approach (Chapter \ref{chap:nonparregclass}),
Equation (\ref{oneonelogodds2}) shows us how to make the necessary
adjustment.  When we have a new case to predict, we do the following:

\begin{itemize}

\item [(a)] Our software has given us an estimate of the left-hand side
of that equation, for $t$ equal to our new data point.

\item [(b)] We know the value that our software has used for its
estimate of $(1-q)/q$, which is $N_0/N_1$.

\item [(c)] Using (a) and (b), we can solve for the estimate of
$f_o(t)/f_1(t)$.

\item [(d)] Now plug the correct estimate of $(1-q)/q$, and the result
of (c), back into (\ref{oneonelogodds2}) to get the proper estimate of
the desired conditional probability.

\end{itemize}

Now what about Problem (b) above, the unbalanced situation?  The easiest
solution is to deliberately use the {\it wrong} value of $q$ in the
above adjustment procedure.

\section{Model Selection in Classification}

Since, as has been emphasized here, classification is merely a special
case of regression, the same issues arise for model selection as in
Section \ref{regmodsel}.  The only new issue is what to do with
nonparametric models.  For instance, with k-Nearest Neighbor estimation,
how do we choose k?

Here the standard technique is again, as in Section \ref{regmodsel}, to split
the data into training and validation sets.  We could fit models with
various values of k to the training set, and then see how well each does
on the validation set.

The same points apply to deciding which predictors to use, and which to
discard, as well as (in the parametric model case) whether to add
polynomial terms and so on.

\section{Optimality of the Regression Function for 0-1-Valued Y
(optional section)}
\label{optimal2}

Remember, our context is that we want to guess Y, knowing X.  Since Y is
0-1 valued, our guess for Y based on X, g(X), should be 0-1 valued too.
What is the best function g()?

Again, since Y and g are 0-1 valued, our criterion should be what will I
call Probability of Correct Classification (PCC):\footnote{This assumes
that our goal is to minimize the overall misclassification error rates,
which in terms assumes equal costs for the two kinds of classification
errors, i.e. that guessing Y = 1 when Y = 0 is no more or no less
serious than the opposite error.}

\begin{equation}
\label{pcc}
\textrm{PCC} = P[Y = g(X)]
\end{equation}

We'll show intuitively that the best rule, i.e. the g() that maximizes
(\ref{pcc}), is given by the function

\begin{equation}
\label{bestg}
g(t) = 
\begin{cases}
0, \textrm{ if } g(t) \leq 0.5 \cr
1, \textrm{ if } g(t) > 0.5 \cr
\end{cases}
\end{equation}

Think of this simple situation:  There is a biased coin, with
\underline{known} probability of heads p.  The coin will be tossed once,
and you are supposed to guess the outcome.  

Let's name your guess q, and let C denote the as-yet-unknown outcome of
the toss (1 for heads, 0 for tails).  Then the probability that you
guess correctly is

\begin{eqnarray}
P(C = q) &=& P(C = 1) q + P(C = 0) (1-q) \\ 
&=& P(C = 1) q + [1 - P(C = 1)] (1-q) \\ 
&=& [2 P(C = 1) - 1] q + 1 - P(C = 1) \\
&=& [2 p - 1] q + 1 - p 
\end{eqnarray}

(That first equation merely accounts for the two cases, q = 1 and q =
0.  For example, if you choose q = 0, then the right-hand side reduces
to P(C = 0), as it should.)

Inspecting the last of the three equations above, we see that if we set
q = 1, then P(C = q), i.e. the probability that we correctly predict the
coin toss, is p.  If we set q to 0, then P(C = q) is 1-p.  That in turn
says that if $p > 0.5$ (remember, p is known), we should set q to 1;
otherwise we should set q to 0.

The above reasoning gives us very intuitive---actually trivial---result:

\begin{quote}
If the coin is biased toward heads, we should guess heads.  If the coin
is biased toward tails, we should guess tails.
\end{quote}

Now returning to (\ref{bestg}), would take P(C = q) above as the
conditional probability P(Y = g(X) $|$ X).  The above coin example says
we should predict Y to be 1 or 0, depending on whether g(X) is larger or
smaller than 0.5.  Then use (\ref{adamsdiscrete}) to complete the proof.


\startproblemset

\oneproblem
Suppose we are interested in documents of a certain type, which we'll
call Type 1.  Everything that is not Type 1 we'll call Type 2, with
a proportion $q$ of all documents being Type 1.  Our goal will be to 
try to guess document type by the presence of absence of a certain word;
we will guess Type 1 if the word is present, and otherwise will guess
Type 2.  

Let $T$ denote document type, and let $W$ denote the event that the word
is in the document.  Also, let $p_i$ be the proportion of documents that
contain the word, among all documents of Type i, i = 1,2.  The event $C$
will denote our guessing correctly.

Find the overall probability of correct classification, $P(C)$, and also
$P(C | W)$.

Hint:  Be careful of your conditional and unconditional probabilities
here.

\oneproblem
We showed that (\ref{oneonelogodds2}) reduces to the logistic model in
the case in which the distribution of $X$ given $Y$ is normal.  Show
that this is also true in the case in which that distribution is
exponential, i.e.

\begin{equation}
f_{X|Y}(t,i) = \lambda_i e^{-\lambda_i t}, ~ t > 0
\end{equation}

% \section{Linear Regression with All Predictors Being Nominal Variables:
% Analysis of ``Variance''}
% \label{anova}
% 
% (Note to readers:  The material in this section is arguably of lesser
% value to computer science.  As such, it can easily be skipped.  However,
% it does provide motivation for our treatment of the log-linear model in
% Section \ref{loglin}.)
% 
% Continuing the ideas in Section \ref{nominal}, suppose in the software
% engineering study they had kept the project size constant, and instead
% of $X^{(1)}$ being project size, this variable recorded whether the
% programmer uses an integrated development environment (IDE).  Say
% $X^{(1)}$ is 1 or 0, depending on whether the programmer uses the
% Eclipse IDE or no IDE, respectively.  Continue to assume the study
% included the nominal Language variable, i.e. assume the study included
% the indicator variables $X^{(2)}$ (C++) and $X^{(3)}$ (Java).  Now all
% of our predictors would be nominal/indicator variables.  Regression
% analysis in such settings is called {\bf analysis of variance} (ANOVA).
% 
% Each nominal variable is called a {\bf factor}.  So, in our software
% engineering example, the factors are IDE and Language.  Note again that
% in terms of the actual predictor variables, each factor is represented
% by one or more indicator variables; here IDE has one indicator variables
% and Language has two.
% 
% Analysis of variance is a classic statistical procedure, used heavily in
% agriculture, for example.  We will not go into details here, but mention
% it briefly both for the sake of completeness and for its relevance to
% Sections \ref{interaction} and \ref{loglin}.  (The reader is strongly
% advised to review Sections \ref{interaction} before continuing.)
% 
% \subsection{It's a Regression!}
% 
% The term {\it analyisis of variance} is a misnomer.  A more appropriate
% name would be {\bf analysis of means}, as it is in fact a regression
% analysis, as follows.
% 
% First, note in our software engineering example we basically are talking
% about six groups, because there are six different combinations of values
% for the triple $(X^{(1)},X^{(2)},X^{(3)})$.  For instance, the triple
% (1,0,1) means that the programmer is using an IDE and programming in
% Java.  Note that triples of the form (w,1,1) are impossible.
% 
% So, all that is happening here is that we have six groups with six
% means.  But that is a regression!  Remember, for variables U and V,
% $m_{V;U}(t)$ is the mean of all values of V in the subpopulation group
% of people (or cars or whatever) defined by U = s.  If U is a continuous
% variable, then we have infinitely many such groups, thus infinitely many
% means.  In our software engineering example, we only have six groups,
% but the principle is the same.  We can thus cast the problem in
% regression terms:
% 
% \begin{equation}
% m_{Y;X}(i,j,k) = E(Y|X^{(1)}=i, X^{(2)}=j, X^{(3)}=k), ~ i,j,k=0,1, j+k
% \leq 1
% \end{equation}
% 
% Note the restriction $j+k \leq 1$, which reflects the fact that j and k
% can't both be 1.
% 
% Again, keep in mind that we are working with means.  For instance,
% $m_{Y;X}(0,1,0)$ is the population mean project completion time for
% the programmers who do not use Eclipse and who program in C++.
% 
% Since the triple (i,j,k) can take on only six values, m can be modeled
% fully generally in the following six-parameter linear form:
% 
% \begin{equation}
% \label{full}
% m_{Y;X}(i,j,k) = \beta_0 + \beta_1 i + \beta_2 j + \beta_3 k
% + \beta_4 ij
% + \beta_5 ik
% \end{equation}
% 
% where $\beta_4$ and $\beta_5$ are the coefficients of two interaction
% terms, as in Section \ref{interaction}.  
% 
% \subsection{Interaction Terms}
% 
% It is crucial to understand the interaction terms.  Without the ij and
% ik terms, for instance, our model would be
% 
% \begin{equation}
% \label{nointeraction}
% m_{Y;X}(i,j,k) = \beta_0 + \beta_1 i + \beta_2 j + \beta_3 k
% \end{equation}
% 
% which would mean (as in Section \ref{interaction}) that the difference
% between using Eclipse and no IDE is the same for all three programming
% languages, C++, Java and C.  That common difference would be $\beta_1$.
% If this condition---the impact of using an IDE is the same across
% languages---doesn't hold, at least approximately, then we would use the
% full model, (\ref{full}).  More on this below.
% 
% Note carefully that there is no interaction term corresponding to jk,
% since that quantity is 0, and thus there is no three-way interaction
% term corresponding to ijk either.  
% 
% But suppose we add a third factor, Education, represented by the
% indicator $X^{(4)}$, having the value 1 if the programmer has a least a
% Master's degree, 0 otherwise.  Then m would take on 12 values, and the
% full model would have 12 parameters: 
% 
% \begin{equation}
% \label{full3}
% m_{Y;X}(i,j,k,l) = \beta_0 + \beta_1 i + \beta_2 j + \beta_3 k + \beta_4 l
% + \beta_5 ij
% + \beta_6 ik
% + \beta_7 il
% + \beta_8 jl
% + \beta_9 kl
% + \beta_{10} ijl
% + \beta_{11} ikl
% \end{equation}
% 
% Again, there would be no ijkl term, as jk = 0.
% 
% Here $\beta_1$, $\beta_2$, $\beta_3$ and $\beta_4$ are called the {\bf
% main effects}, as opposed to the coefficients of the interaction terms,
% called of course the {\bf interaction effects}.
% 
% The no-interaction version would be
% 
% \begin{equation}
% \label{noint3}
% m_{Y;X}(i,j,k,l) = \beta_0 + \beta_1 i + \beta_2 j + \beta_3 k + \beta_4 l
% \end{equation}
% 
% \subsection{Now Consider Parsimony}
% 
% In the three-factor example above, we have 12 groups and 12 means.  Why
% not just treat it that way, instead of applying the powerful tool of
% regression analysis?  The answer lies in our desire for parsimony, as
% noted in Section \ref{overfit}.
% 
% If for example (\ref{noint3}) were to hold, at least approximately, we
% would have a far more satisfying model.  We could for instance then talk
% of ``the'' effect of using an IDE, rather than qualifying such a
% statement by stating what the effect would be for each different
% language and education level.  Moreover, if our sample size is not very
% large, we would get more accurate estimates of the various subpopulation
% means, once again due to bias/variance tradeoff.
% 
% Or it could be that, while (\ref{noint3}) doesn't hold, a model with only
% two-way interactions,
% 
% \begin{equation}
% \label{2way}
% m_{Y;X}(i,j,k,l) = \beta_0 + \beta_1 i + \beta_2 j + \beta_3 k + \beta_4 l
% + \beta_5 ij
% + \beta_6 ik
% + \beta_7 il
% + \beta_8 jl
% + \beta_9 kl
% \end{equation}
% 
% does work well.  This would not be as nice as (\ref{noint3}), but it
% still would be more parsimonious than (\ref{full3}).
% 
% Accordingly, the major thrust of ANOVA is to decide how rich a model is
% needed to do a good job of describing the situation under study.  There
% is an implied hierarchy of models of interest here:
% 
% \begin{itemize}
% 
% \item the full model, including two- and three-way interactions,
% (\ref{full3})
% 
% \item the model with two-factor interactions only, (\ref{2way})
% 
% \item the no-interaction model, (\ref{noint3})
% 
% \end{itemize}
% 
% Traditionally these are determined via hypothesis testing, which
% involves certain partitionings of sums of squares similar to
% (\ref{sumsq}).  (This is where the name {\it analysis of variance} stems
% from.)  The null distribution of the test statistic often turns out to
% be an F-distribution.  Of course, in this book, we consider hypothesis
% testing inappropriate, preferring to give some careful thought to the
% estimated parameters, but it is standard.  Further testing can be done
% on individual $\beta_1$ and so on.  Often people use simultaneous
% inference procedures, discussed briefly in Section \ref{simultancis} of
% our chapter on estimation and testing, since many tests are performed.
% 
% \subsection{Reparameterization}
% 
% Classical ANOVA uses a somewhat different parameterization than that
% we've considered here.  For instance, consider a single-factor setting
% (called {\bf one-way ANOVA}) with three levels.  Our predictors are then
% $X^{(1)}$ and $X^{(2)}$.  Taking our approach here, we would write
% 
% \begin{equation}
% m_{Y;X}(i,j) = \beta_0 + \beta_1 i + \beta_2 j 
% \end{equation}
% 
% The traditional formulation would be
% 
% \begin{equation}
% \mu_i = \mu + \alpha_i, ~ i = 1,2,3
% \end{equation}
% 
% where
% 
% \begin{equation}
% \mu = \frac{\mu_1+\mu_2+\mu_3}{3}
% \end{equation}
% 
% and 
% 
% \begin{equation}
% \alpha_i = \mu_i - \mu
% \end{equation}
% 
% Of course, the two formulations are equivalent.  It is left to the
% reader to check that, for instance, 
% 
% \begin{equation}
% \mu = \beta_0 + \frac{\beta_1+\beta_2}{2}
% \end{equation}
% 
% There are similar formulations for ANOVA designs with more than one
% factor.
% 
% Note that the classical formulation overparameterizes the problem.  In
% the one-way example above, for instance, there are four parameters
% ($\mu$, $\alpha_1$, $\alpha_2$, $\alpha_3$) but only three groups.
% This would make the system indeterminate, but we add the constraint
% 
% \begin{equation}
% \sum_{i=1}^3 \alpha_i = 0
% \end{equation}
% 
% Equation (\ref{betahat}) then must make use of {\bf generalized matrix
% inverses}.
% 
% \section{Clustering}
% 
% HERE ONE **REALLY* HAS TROUBLE 
% WITH "WHAT DOES IT MEAN?:

% pdf("CARTBound.pdf")
% plot(c(0,1),c(0,1),type="n",xlab="s",ylab="t")
% lines(c(0,0.62),c(0.8,0.8))
% lines(c(0.62,1),c(0.58,0.58))
% lines(c(0.62,0.62),c(0.58,0.8))
% dev.off()
% 
% 
% lines(c(0.62,0.62),c(0,1))

