\documentclass{beamer}
\usepackage{graphicx}
\usepackage{url}
\usepackage{fancyvrb}
\usepackage{relsize}
% \usepackage[pdftex]{color}

\usepackage{CJKutf8}

\mode<presentation>
{ \usetheme{Darmstadt} }

\title{Are They the Best and the Brightest? \\
Analysis of Employer-Sponsored Tech Immigrants}

\author{
Norm Matloff \\
Department of Computer Science \\
University of California at Davis
}

\date{Institute for the Study of International Migration \\
Georgetown University \\
March 18, 2011}
 
\begin{document} 

\begin{frame}
\titlepage

\end{frame}

\begin{frame}
\frametitle{The Setting}

\pause

\begin{itemize}

\item ...``[restrictive U.S. immigration policy is] driving away the 
world's \textcolor{red}{best and brightest}''---Bill Gates, 2007
% International Educator, May/June 2007
\pause

\item ``We should not [send our] \textcolor{red}{bright and talented}
international students...to work for our competitors abroad upon
graduation''--NAFSA (Nat. Assoc. of Foreign Student Advisers)
% http://www.nafsa.org/public_policy.sec/international_student_1/immigration_reform_issues/comprehensive_immigration_3
\pause

\item ``...we should be stapling a green card to the diploma of any
foreign student who earns an advanced degree at any U.S. university...
\textcolor{red}{The world’s best brains are on sale}. Let’s buy
more!''---{\it New York Times} columnist Tom Friedman, 2009
% NYT, June 27, 2009
\pause

\item Industry wants more H-1B work visas, and fast-track green cards
for STEM foreign students.

\end{itemize}

\end{frame}

\begin{frame}
\frametitle{Questions}

\begin{itemize}

\item We all support the immigration of outstanding talents, the
innovative, the ``game changers,'' etc.
\pause

\item But are most of those sponsored by the tech industry of that caliber?
\pause

\item How do rates of top foreign talent vary from employer to employer?
\pause

\item What are rates of top foreign talent among the main nationalities,
i.e. Chinese and Indian?

\end{itemize}

\end{frame}

\begin{frame}
\frametitle{Previous Work}
\pause

\begin{itemize}
 
\item (North, 1995) found foreign engineering PhD students were
concentrated in lower-ranked universities.
\pause

\item (Hunt, 2009, 2011) considered  general immigrants, not just STEM.
Found: 
\pause

   \begin{itemize}

   \item immigrants paid less (but Europeans paid more)
   \pause
   
   \item immigrants patent at rates $\leq$ natives
   \pause

   \item immigrants had more research pubs and higher rates of
   entrepreneurship

   \end{itemize}
\pause

\item Not much else.
\pause

\item I'm considering only studies that provide comparison to
Americans.\footnote{American = U.S. citizen (native or naturalized) or
green card holder.} 
\pause

\item 
E.g. reporting plain \#s of immigrant patents is NOT meaningful.
\pause
(Lots of immigrants $\Rightarrow$ lots of immigrant patents.)

\end{itemize}

\end{frame}

\begin{frame}
\frametitle{Our Aproach}
\pause

\begin{itemize}

\item Focus:
\pause

   \begin{itemize}

   \item Mixing quite disparate groups (all STEM fields) ignores
   interaction effects, thus clouding issues.
   \pause

   \item So, I focus on computer science and electrical engineering.
   \pause

   \item CS/EE forms the bulk of H-1Bs.
   \pause

   \item For CS/EE, I know ``where the bodies are buried.''
   \pause

   \end{itemize}

\item Criteria for ``best and brightest'':
\pause

   \begin{itemize}

   \item Higher salaries than Americans. 
   \pause

   \item Higher \% of awards than Americans.
   \pause

   \item Higher \% of patents than Americans.

   \end{itemize}

\end{itemize}

\end{frame}

\begin{frame}
\frametitle{Criteria NOT Used}
\pause

\begin{itemize}

\item Numbers of research publications:
\pause

   \begin{itemize}
    
   \item ``Deans can count but they can't read.''
   \pause

   \item Many researchers are ``CV builders.''
   \pause

   \item Quantity $\neq$ quality.
   \pause

   \end{itemize}

\item Entrepreneurship:
\pause

   \begin{itemize}

   \item Entrepreneurship $\neq$ innovation/U.S. jobs.
   \pause

   \item Saxenian (1999) found that 36\% of Chinese-immigrant
   firms were in ``Computer [PC] Wholesaling.''
   \pause

   \item Many Indian-immigrant firms are in the outsourcing business.

   \end{itemize}

\end{itemize}

\end{frame}

\begin{frame}
\frametitle{Wage Issues}
\pause

\begin{itemize}

\item Foreign workers exploitable, esp. if sponsored for green card. 
\pause

\item Underpayment found to be 15-20\% in (Matloff, 2003) and 33\% in
(Ong, 1997).  (Separate from age issues.)
\pause

\item Underpayment due to {\bf loopholes} in {\it prevailing
wage}.\footnote{Similar loopholes for legal definition of ``actual
wage.''}
\pause

\item Congressionally-commissioned employer surveys, (NRC 2001) and
(GAO 2003), found many employers admitting to paying H-1B workers less
than comparable Americans.
\pause

\item GAO even noted role of loopholes:  

\begin{quote}
... [employers] hired H-1B workers in part because these workers 
would often accept lower salaries...however, these employers said 
they never paid H-1B workers less than the required wage. 
\end{quote}

\end{itemize}

\end{frame}

\begin{frame}
\frametitle{Solutions to Wage Issues}
\pause

\begin{itemize}

\item Analyses based on wages must account for the underpayment of
H-1Bs. 
\pause

\item Where possible restriction attention to green card holders (LPRs) and
nat. citizens.
\pause

\item Artificially ``raise'' H-1B salaries by factor 1.2.

\end{itemize}

\end{frame}

\begin{frame}
\frametitle{Nationality Issues}
\pause

Majority of tech foreign workers are Indians and Chinese. 
\pause

\begin{itemize}

\item Among computer-related H-1Bs, 64.8\% Indian, 8.2\% Chinese
(Filipinos third, at 2.3\%)  (INS, 2001).
\pause

\item In 2009 employer apps for worker green cards, 59.0\% were for
Indians and 7.5\% for Chinese.
\pause

\item Among those who (ever) came to U.S. as foreign students in CS/EE
and were working in CS/EE as of 2003, 23.2\% were Chinese and 
27.2\% were Indian.

\end{itemize}

\end{frame}

\begin{frame}
\frametitle{Interesting Time Trend}
\pause

CSEE EB green cards apps trend $\downarrow$ for Chinese, $\uparrow$
for Indians:
\pause

\begin{tabular}{|r|r|r|}
\hline
year  & China & India \\ \hline
\hline
2005 & 0.134 & 0.444 \\ \hline
2006 & 0.103 & 0.501 \\ \hline
2007 & 0.097 & 0.515 \\ \hline
2008 & 0.080 & 0.569 \\ \hline
2009 & 0.075 & 0.590 \\ \hline
\end{tabular}

\end{frame}

\begin{frame}
\frametitle{First Wage Analysis:  NSCG Data}
\pause

\begin{itemize}

\item National Survey of College Graduates (2003) (Hunt's data)
\pause 

\item My focus: Degree {\it and} (nonacademic) job in CS/EE.
\pause

\item My focus: Green card holders and citizens only.
\pause

\item My focus:  Imms. came to U.S. as foreign students (F-1).

\end{itemize}

\end{frame}

\begin{frame}
\frametitle{NSCG Salaries: Results}

\pause

Regression model; response variable is salary.
\pause

\medskip

\begin{columns}

  \begin{column}{0.5\textwidth}
   \begin{tabular}{|r|r|}
   \hline
   factor & beta, marg. err.  \\ \hline
   \hline
   const. & -3272 {\footnotesize $\pm$ 18383} \\ \hline
   age & 3400 {\footnotesize $\pm$ 863} \\ \hline
   age $\times$ age & -34 {\footnotesize $\pm$ 10} \\ \hline
   MS & 8809 {\footnotesize $\pm$ 2173} \\ \hline
   PhD & 22495 {\footnotesize $\pm$ 4512} \\ \hline
   highCOL & 8725 {\footnotesize $\pm$ 1918} \\ \hline
   origF1 & 808 {\footnotesize $\pm$ 3019} \\ \hline
   \end{tabular}
  \end{column}
  
  \begin{column}{0.5\textwidth}
  \pause
  \begin{itemize}
  \item note negative quadratic age effect
  \pause
  \item highCOL = high cost-of-living region
  \pause
  \item origF1 = came to U.S. as foreign student
  \pause
  \item no overall evidence of ``best and brightest''
  \end{itemize}
  \end{column}

\end{columns}

\end{frame}

\begin{frame}
\frametitle{NSCG Salaries: Results, contd.}

Separate out the Indians and Chinese (``ICs''). 
\pause

\medskip

\begin{columns}

  \begin{column}{0.5\textwidth}
   \begin{tabular}{|r|r|}
   \hline
   factor & beta, marg. err.  \\ \hline
   \hline
   const. & -2640 {\footnotesize $\pm$ 18429} \\ \hline
   age & 3369 {\footnotesize $\pm$ 865} \\ \hline
   age $\times$ age & -33 {\footnotesize $\pm$ 10} \\ \hline
   MS & 9948 {\footnotesize $\pm$ 2177} \\ \hline
   PhD & 22667 {\footnotesize $\pm$ 4509} \\ \hline
   highCOL & 8692 {\footnotesize $\pm$ 1917} \\ \hline
   origF1nonIC & 4479 {\footnotesize $\pm$ 3847} \\ \hline
   origF1chn & -6190 {\footnotesize $\pm$ 5632} \\ \hline
   origF1ind & -978 {\footnotesize $\pm$ 5571} \\ \hline
   \end{tabular}
  \end{column}
  
  \begin{column}{0.5\textwidth}
  \pause
  \begin{itemize}
  \item non-ICs paid $>$ avg., about 0.5 MS effect
  \pause
  \item Chinese paid $<$ avg. about 2/3 MS effect
  \end{itemize}
  \end{column}

\end{columns}

\end{frame}

\begin{frame}
\frametitle{Second Wage Analysis:  PERM Data}

\pause

\begin{itemize}

\item DOL files of employer applications for green card (CS/EE). 
% http://www.flcdatacenter.com/CasePerm.aspx
\pause

\item Enables analysis by employer and nationality.
\pause

\item Accounts for region via prevailing wage.
\pause

\item Lacks data on education, age.

\end{itemize}

\end{frame}

\begin{frame}
\frametitle{PERM Analysis}

\pause

\begin{itemize}

\item I calculated the median wage ratio:

$$
\textrm{WR = median of} ~~~
   \frac{\textrm{actual wage}}{\textrm{emp. claimed prev. wg.}}
$$

\pause

\item By law, must have $WR \geq 1$.
\pause

\item But, denominator too small by factor of 1.15 to 1.33 (prev. wg.
def. loopholes).
\pause

\item ``Best and brightest'' salary premiums (U.S. workers):
\pause

   \begin{itemize}

   \item New Stanford CS grads earn 37\% more than average CS.  
   \pause

   \item 20 years after graduation, Stanford grads (general) earn 28\%
   more than San Jose State Unv. grads.
   \pause

   \item Grads (general) of most selective schools have starting salaries
   45\% more than least selective.
   \end{itemize}

\pause

\item So, only (median) WR values higher than, say 1.25, indicate a firm is
hiring mainly the ``best and brightest'' foreign workers.

\end{itemize}

\end{frame}

\begin{frame}
\frametitle{Overall PERM Results}

Median WR values (SE=sw. engineers: EE=elec. engineers):

\medskip

\begin{columns}

  \begin{column}{0.5\textwidth}
  \begin{tabular}{|r|r|}
  \hline
  group & med. WR \\ \hline 
  \hline
  SE & 1.01 \\ \hline 
  EE & 1.00 \\ \hline 
  Chinese SE & 1.02 \\ \hline 
  Indian SE & 1.01 \\ \hline 
  Chinese EE & 1.01 \\ \hline 
  Indian EE & 1.01 \\ \hline 
  \end{tabular}
  \end{column}
  \pause

  \begin{column}{0.5\textwidth}

  \begin{itemize}

  \item Almost no variation.
  \pause

  \item Shows that most employers use Prev. Wg., not Actual Wg.
  (see previous footnote).
  \pause
  
  \item No overall evidence of ``best and brightest.''

  \end{itemize}
  
  \end{column}

\end{columns}

\end{frame}

\begin{frame}
\frametitle{PERM Results by Firm}
\pause

Median WR for some prominent firms with large numbers of PERM entries:
\pause

\bigskip

\begin{columns}

   \begin{column}{0.5\textwidth}
   \begin{tabular}{|r|r|r|}
   \hline
   firm & WR & n \\ \hline
   \hline
   HP & 1.20 & 243 \\ \hline
   Microsoft & 1.18 & 4039 \\ \hline
   Intel & 1.14 & 1465 \\ \hline
   Oracle & 1.13 & 830 \\ \hline
   Google & 1.12 & 690 \\ \hline
   eBay & 1.05 & 118 \\ \hline
   Cisco & 1.04 & 1135 \\ \hline
   Motorola & 1.00 & 848 \\ \hline
   Qualcomm & 1.00 & 268 \\ \hline
   \end{tabular}
   \pause
   \end{column}

   \begin{column}{0.5\textwidth}

   \begin{itemize}
   
   \item Considerable variation among firms.  \pause

   \item But remember, different firms use different methods for
   calculating prev. wg. \pause
   
   \item A few firms pay a 10-15\% premium.  \pause
   
   \item No firm has WR high enough to qualify as hiring ``best and
   brightest.''
   
   \end{itemize} \end{column}

\end{columns}

\end{frame}

\begin{frame}
\frametitle{ACM Dissertation Awards}

\pause

\begin{itemize}

\item Assoc. for Computing Machinery, the main professional CS body
\pause

\item 58 awards since 1982
\pause

\item no direct data on foreign/domestic; names used as proxies
\pause

\item 25 of the 58 foreign, slightly underrepresented, 43\%, similar to
42\% figure for foreign among CS PhDs overall (NSCG)
\pause

\item no evidence that the foreign students are outperforming the
domestic ones

\end{itemize}

\end{frame}

\begin{frame}
\frametitle{ACM Awards, contd.}
\pause

Of 58 awards, 2 from China, 8 from India.
\pause

\medskip

\begin{tabular}{|r|r|r|}
\hline
nationality & $\%$ of awardees & $\%$ of CS PhDs \\ \hline
\hline
China & 3.5\% $\pm$ 4.8\% & 28.6\% $\pm$ 8.1\% \\ \hline
India & 13.8\% $\pm$ 9.1\% & 19.0\% $\pm$ 7.0\% \\ \hline
\end{tabular}

\pause

\medskip

\begin{itemize}

\item Chinese award rate much lower than average
\pause

\item Indian rate could be about average

\end{itemize}

\end{frame}

\begin{frame}
\frametitle{Patents:  NSCG}

\pause

Regression analysis; response variable is number of patent apps filed:
\pause

\medskip

\begin{columns}

  \begin{column}{0.5\textwidth}
   \begin{tabular}{|r|r|}
   \hline
   factor & beta $\pm$ marg. err.  \\ \hline
   \hline
   const. & 0.11 {\small $\pm$ 0.31} \\ \hline
   age & 0.00 {\small $\pm$ 0.01} \\ \hline
   MS & 0.25 {\small $\pm$ 0.15} \\ \hline
   PhD & 2.65 {\small $\pm$ 0.29} \\ \hline
   origF1 & 0.04 {\small $\pm$ 0.19} \\ \hline
   \end{tabular}
  \end{column}
  
  \begin{column}{0.5\textwidth}
  \pause
  \begin{itemize}
  \item No overall ``best and brightest'' effect.
  \pause
  \item PhD only major effect.
  \end{itemize}
  \end{column}

\end{columns}

\end{frame}

\begin{frame}
\frametitle{NSCG patents, contd.}

\pause

Separate out the Indians and Chinese (``ICs''): 
\pause

\medskip

\begin{columns}

  \begin{column}{0.5\textwidth}
   \begin{tabular}{|r|r|}
   \hline
   factor & beta $\pm$ marg. err.  \\ \hline
   \hline
   const. & 0.11 {\small $\pm$ 0.31} \\ \hline
   age & 0.00 {\small $\pm$ 0.01} \\ \hline
   MS & 0.27 {\small $\pm$ 0.15} \\ \hline
   PhD & 2.65 {\small $\pm$ 0.29} \\ \hline
   origF1nonIC & 0.35 {\small $\pm$ 0.25} \\ \hline
   origF1China & -0.50 {\small $\pm$ 0.36} \\ \hline 
   origF1India & -0.11 {\small $\pm$ 0.33} \\ \hline
   \end{tabular}
  \end{column}
  
  \begin{column}{0.5\textwidth}
  \pause
  \begin{itemize}
  \item Chinese MS app count much $<$ than American Bachelor's.
  \pause
  \item Non-ICs' Bachelor's count higher than Americans' MS. 
  \end{itemize}
  \end{column}

\end{columns}

\end{frame}

\begin{frame}
\frametitle{Why Did the Chinese Workers Fare Poorly?}

\pause

First possible factor:  Rote-memory learning culture inhibits
creativity.

\pause

\begin{itemize}

\item Well recognized, with its own Chinese term,
% \begin{CJK}{UTF8}{gbsn}填鸭\end{CJK}
% \begin{CJK}{UTF8}{bsmi}子\end{CJK}---tian yazi, ``stuff the duck.''
\pause

\item A common complaint among prominent Chinese academics, e.g. SUNY
Stony Brook's C.N. Yang. \pause

\item Governments of China, Japan, S. Korea and Taiwan have all tried to
remedy this.

\end{itemize}

\end{frame}

\begin{frame}
\frametitle{Chinese Case, contd.}

\pause

Second possible factor:  The Chinese workers may be handicapped by language
issues.
\pause
This seems to not be a strong factor.

\pause

\begin{itemize}

\item Foreign students from China in the last 10-15 years have tended to
have very good English. \pause

\item The ACM Dissertation Awards go mostly to students at the elite
schools, which have very stringent English requirements for admission.
\pause

\item E.g. MIT, Harvard and Columbia require a TOEFL minimum score of
109/120 for admission. \pause

\item The tech industry is famously meritocratic for engineering (not
managerial) workers.  If you produce, you are rewarded.  English is not
a major issue. \pause

\item Logistic regression analysis on the PUMS census data shows that among
immigrant Chinese, English skill has no impact on the probability of
earning a high-level salary ($>$ \$150K). 
% http://toeflnow.com/toefl-score

\end{itemize}

\end{frame}

\begin{frame}
\frametitle{Is There a CS/EE Labor Shortage?}

\pause

\begin{itemize}

\item Even if the foreign workers are not especially talented, their
large numbers might be justified if there were a labor shortage.
\pause

\item But among the many government and other studies, none (other than
by the industry) has ever shown a shortage.  
\pause

\item Salaries (adjusted for inflation) have been flat, 
counterindicating a shortage.  
\pause

\item Employers still very picky in hiring, again counterindicating a
shortage.

\end{itemize}

\end{frame}

\begin{frame}
\frametitle{Internal Brain Drain}

\pause

A surplus of workers is causing an {\bf internal brain drain}.
\pause

\begin{itemize}

\item Workers become less employable around age 35.\footnote{Nice graph
in GAO report, BTW.} 
\pause

\item National Science Foundation advocated H-1B with explicit goal of
holding down PhD salaries.
\pause
Forecast (correctly) that stagnant wages would then drive American 
students away from PhD.  
\pause 

\item Stagnant CS/EE salaries at all levels discourage young people from
entering the field.
\pause

\item Post doc program, fueled by H-1B, makes lab science careers
\underline{extremely} unattractive to young people.
\pause

\item ``Innovation'' is the buzzword {\it de jour}, and it is U.S.' only
comparative advantage.  Yet the system is wasting that advantage.

\end{itemize}

\end{frame}

\begin{frame}
\frametitle{Discussion:  Policy Implications, H-1B}

\pause

\begin{itemize}

   \item Employer claims that H-1B visas are needed to bring in
   outstanding talents are not borne out. \pause

   \item ``We should return the H-1B visa to its original intent,
   bringing in the best and the brightest''--Rep. Zoe Lofgren, House
   hearing, 1998.
   \pause

   \item Should prioritize granting of H-1B requests by wage level.
   \pause

   \item Should adopt Durbin-Grassley definition of prevaling wage.

\end{itemize}

\end{frame}

\begin{frame}
\frametitle{Discussion:  Policy Implications, Green Cards}

\pause

\begin{itemize}

   \item No best/brightest trend was found here among
   foreign students. \pause  
   
   \item Thus a blanket green card program for STEM foreign students
   would be unwarranted.  \pause

   \item Currently have long waits for green cards in EB-3 category---the
   wrong group to offer a remedy, as it is exactly the one for the least
   talented workers.  \pause

   \item Should transfer much of the EB-3 quota to EB-2.

\end{itemize}

\end{frame}

\end{document} 

\begin{frame}
\frametitle{Final Remarks}
\pause

\begin{itemize}

\item Most of the sponsored foreign workers appear to be of ordinary
talent.
\pause

\item But again, some are indeed truly outstanding talents. 
\pause

\item We should facilitate the immigration of such talents.
\pause

\item Recently there has been some concern about long green card waits
for employer-sponsored workers.  However, for PhDs, who have their own
category, the wait continues to be short.

\end{itemize}
\end{frame}

\begin{frame}
\frametitle{Resources}

These slides, and the R programming code used to compile the statistics,
are available at \url{http://heather.cs.ucdavis.edu/BGIT.html}

\end{frame}
