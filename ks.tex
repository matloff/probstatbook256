
\subsection{Kolmogorov-Smirnov Confidence Bands}
\label{kolsmi}

Again consider the problem above, in which we were assessing the fit of
a exponential model.  In line with our major point that confidence
intervals are far superior to hypothesis tests, we now present {\bf
Kolmogorov-Smirnov confidence bands}, which work as follows.

Since this method relies on cdfs, recall  
notion of the {\bf empirical distribution function} (ecdf), Section
\ref{ecdfsec}.
It is a sample estimate of a cdf, defined to be the proportion
of $X_i$ that are below t in the sample.  Graphically, $\widehat{F}_X$
is a step function, with jumps at the values of the $X_i$.

What Kolmogorov-Smirnov does is form a {\bf confidence band} around the
empirical cdf of a sample.  The basis for this is that the distribution
of

\begin{equation}
\label{definem}
M = \max_{-\infty < t \infty} |\widehat{F}_X(t) - F_X(t)|
\end{equation}

{\bf is the same for all distributions having a density}.  This 
fact (whose proof is related to the general method for simulating random
variables having a given density, in Section \ref{genrannumgen})
tells us that, without knowing anything about the distribution of
X, we can be sure that M has the same distribution.  And it turns out
that

\begin{equation}
\label{m95}
F_M(1.358 n^{-1/2}) \approx 0.95
\end{equation}

Define ``upper'' and ``lower'' functions

\begin{equation}
U(t) = \widehat{F}_X(t) + 1.358 n^{-1/2}, ~~
L(t) = \widehat{F}_X(t) - 1.358 n^{-1/2}
\end{equation}

So, what (\ref{definem}) and (\ref{m95}) tell us is

\begin{equation}
0.95 \approx P \left (\textrm{the curve } F_X \textrm{ is entirely between U
and L} \right ) 
\end{equation}

So, the pair of curves, (L(t), U(t)) is called a a {\bf 95\% confidence
band} for $F_X$. 

Now suppose we wish to see how well, say, the gamma distribution family
fits our application.  If the band is very wide, we know we really don't
have enough data to decide much about the distribution of X.  But if the
band is narrow but some member of the family is in the band or is close
to it, we would probably decide that the model is a good one.  Once
again, we should NOT pounce on tiny deviations from the model.

Warning:  The Kolmogorov-Smirnov procedure available in
the R language performs only a hypothesis test, rather than forming a
confidence band.  In other words, it simply checks to see whether a
member of the family falls within the band.  This is not what we want,
because we may be perfectly happy if a member is only {\it near} the
band.


