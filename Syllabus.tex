\input ../GenCourseInfo/Front1.tex  
\input TitleDate132.tex
\input ../GenCourseInfo/Front2.tex

% \section{Important Dates and Deadlines}

\begin{itemize}

\item formation of Homework Groups: January 8 (in disc. sec.)

\item ordinary Quizzes:  weekly

\item Group Quiz:  March 10

\item Group Project due:  March 19

\end{itemize}

\section{Required Course Materials}

\subsection{Textbook}
\label{text}

The textbook is my open source book, available at
\url{http://heather.cs.ucdavis.edu/~matloff/132/PLN/ProbStatBookW16ECS132.pdf}.

{\bf You are required to print your own hard copy.} It is required that
you have a hard copy of the book, as it will be used during Quizzes.

You can print the book anywhere, say at Copyland on G Street,
but you may find it's cheaper elsewhere.  Cost,
including binding, should be between \$20 and \$30.  One student said
she got her book from Digital Copy Printing on 3rd St. for just \$16.  (Note:
When you talk to a vendor on this, be sure to distinguish between {\it
pages} and {\it sheets}.)  Just bring the store a copy of the correct
PDF file, say on a memory stick.

{\bf PLEASE NOTE:}  Pagination matters!  Quizzes (which are open book)
will refer to specific pages in the book.  So you need
\underline{exactly} the particular file stated above.  

We will cover the entire book.  Note, though, that it is actually an
abridged version of 
\url{http://heather.cs.ucdavis.edu/~matloff/132/PLN/ProbStatBook.pdf},
shortened by removing chapters on advanced material.  I continually
revise the long version, and you may find that sometimes the revised
material is helpful, but it is NOT required.

\subsection{R Manual}

This is not actually required, but unless you already know R well,
you'll find it very useful:

\url{http://heather.cs.ucdavis.edu/~matloff/132/NSPpart.pdf}.  This part
of a VERY rough and partial draft of that my R book.\footnote{{\it The
Art of R Programming}, NSP, 2011.} It is only about 50\% complete, has
various errors, and presents a number of topics differently from the
final version, but should be useful in R work for this class.  

Note that there is a short R tutorial at the end of our textbook.  Read
this first.

\section{What Is This Course?}

This is intended to be a course in probability and statistics similar to
STA 131A and MAT 135A.  Like those courses, ECS 132 is mathematical in
nature,  but with the hugely important difference that it is tailored to
{\it computer science students and computer science applications}.  (Another
important difference is that it covers both probability and statistics.)

Starting in Fall 2009, ECS 132 has been required for CSE majors (and an
elective for CS majors), instead of the old requirement to take STA 131A
or MAT 135A.  This change was made (a) because some CS faculty thought
an in-house course would be better, and (b) because ABET, the
engineering accreditation agency also prefers that the course be
in-house.  Most engineering CS programs at UC campuses teach such
a course in-house.

In addition to the CS applications setting, some other distinguishing
features of this course include:

\begin{itemize}

\item By interweaving the theory with real-world applications, you will
get a much better \underline{practical} understanding of probability
and statistics.

\item You will get some introduction to some methodologies that are
``hot'' in the business world today, notably {\bf data and text mining,
predictive analytics and machine learning}.  These and related
techniques form the very heart of the technology in search engines like
those of Google, and in Big Data applications.  See the excellent {\it
New York Times} article, at
\url{http://www.nytimes.com/2009/08/06/technology/06stats.html}.

There are many other tech firms, big and small, that make extensive use
of this material.  There are even data mining contests, with big prizes
(thousands and even millions of dollars)---see \url{www.kaggle.com}.

\item Your ability to put the theory into practice will be greatly
enhanced by our use of the R statistical programming language.  R is the
standard real-world statistical computational tool in use today.  (Some
of you may have taken STA 32, which uses a bit of R; our usage will go
much further than that level.)

In fact, Google uses R extensively, and it has its own R coding style
guidelines
(\url{http://google-styleguide.googlecode.com/svn/trunk/google-r-style.html}).
They're not to my taste, but you can see that R is a big deal at Google.

There was a nice {\it New York Times} article on R; see
\url{http://www.nytimes.com/2009/01/07/technology/business-computing/07program.html}.

\item Probability and statistics play major roles in our daily lives, in
everything from Lake Tahoe casinos to buying insurance to voting in
elections.  Understanding these concepts enhances our lives.  ECS 132,
as a more practical, data-oriented course, better achieves this goal.

\end{itemize}

\section{Workload}

There will be approximately four assignments, consisting of mathematical
work plus some light programming work.  The math will be
intellectually similar in spirit to ``word problems'' in calculus 
(``Water is flowing into a conical tank with a 30 degree angle from the
vertical, at a rate of...with leakage rate of...'').

{\bf IN ORDER TO ACHIEVE A DECENT GRADE, PLAN TO SPEND SIGNIFICANT TIME
ON CAREFUL READING OF THE TEXT, SEVERAL HOURS PER WEEK.}

All in all, the number of hours per week you'll put in should be similar
to something like ECS 60.  Note, though, that much of this will be Group
work.

\section{Prerequisites}

The required background is:

\begin{itemize}

\item 
$\frac{d}{dt} \sin^2(t) = 2 \sin(t) \cos(t)$, 
\hspace{0.25in}
$\int_{0}^{\infty} \lambda e^{-\lambda t} ~ dt$ = 1,
\hspace{0.25in}
$\sum_{i=0}^{\infty} p^i = \frac{1}{1-p} ~~ (|p| < 1)$

Derivatives, integrals, infinite series.

\item 
$
\left (         
\begin{array}{cc}
a & b \\
c & d
\end{array}
\right )^{-1}     
\left (         
\begin{array}{c}
e \\
f
\end{array}
\right )
$

Basic matrix operations, i.e. addition, multiplication and inverse.

\item {\tt if (n > x+y) z = 168;}

Reasonable programming and debugging skill; basic awareness of the
concepts of bits/bytes, memory addresses and data structures; experience
in writing code to read and write files.

{\bf PLEASE NOTE:}  This is primarily a {\bf MATH} course,  Though we
will do R programming throughout the course, especially in the Quizzes,
{\bf you cannot do the R well without strong insight and intuition into
the math,} no matter how good you are at programming and no matter how
well you know R.  

\end{itemize}

The R language and \LaTeX word processing software that we will use in
this class are available for every major platform---Linux, Macs and
Windows.  However, {\bf I strongly prefer to that you run on a
Unix-family system, either Linux or a Mac.  as that would make it easier
for me to help you fix your bugs}.  So, a basic background in Unix at
the level of usage of ECS 40 would be helpful but not required.  (See
also Section \ref{machines} below.)

{\bf Prior background in R and \LaTeX themselves is NOT required, nor is
any background in probability and statistics.}

\section{Post-STA 131A Courses}

After successfully completing this course, you will be qualified to take
any course which has STA 131A as a prerequisite, such as STA 131B
(mathematical statistics) or MAT 135B (stochastic processes).

\input ../GenCourseInfo/Machines.tex
\input ../GenCourseInfo/LaTeX.tex
\input ../GenCourseInfo/SyllabusCore.tex


