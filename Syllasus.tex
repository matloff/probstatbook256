\input ../GenCourseInfo/Front1.tex 
\input TitleDate132.tex
\input ../GenCourseInfo/Front2.tex

% \section{Important Dates and Deadlines}

\begin{itemize}

\item formation of Homework groups: January 11

\item Midterm Exam:  March 8

\item Group Quiz:  March 15

\item Project due:  March 22

\end{itemize}

\section{Required Course Materials}

\subsection{Textbook}
\label{text}

The textbook is my open source book, available at
\url{http://heather.cs.ucdavis.edu/~matloff/132/PLN/ECS132Winter2012.pdf}.

We will cover the entire book.  Note, though, that it is actually an
abridged version of
\url{http://heather.cs.ucdavis.edu/~matloff/132/PLN/ProbStatBook.pdf},
shortened by removing chapters on advanced material.  I continually
revise the long version, and you may find that sometimes the revised
material is helpful, but it is NOT required.

You can print the book in town, say at Copyland on G Street and Davis
Copy Shop on 3rd St., but you may find it's cheaper at Costco or the
like.  Cost, including binding, should be between \$20 and \$30.

It is required that you have a hard copy of the book, as it will be used
during Tests.

The R manual, at
\url{http://heather.cs.ucdavis.edu/~matloff/145/PLN/RMaterials/145R.pdf}
is not actually required, but unless you already know R well, you'll
find it very useful:


\section{What Is This Course?}

This is intended to be a course in probability and statistics similar to
STA 131A and MAT 135A.  Like those courses, ECS 132 is mathematical in
nature,  but with the hugely important difference that it is tailored to
computer science students and computer science applications.  (Another
important difference is that it covers both probability and statistics.)
Starting in Fall 2009, ECS 132 has been required for CSE majors, instead
of the old requirement to take STA 131A or MAT 135A.  

In addition to the CS applications setting, some other distinguishing
features of this course include:

\begin{itemize}

\item By interweaving the theory with real-world applications, you will
get a much better \underline{practical} understanding of probability
and statistics.

\item You will get some introduction to some methodologies that are
``hot'' in the business world today, notably {\bf data and text mining}.
These and related techniques form the very heart of the technology in
search engines like those of Google, and in many other Artificial
Intelligence applications.  See the excellent {\it New York Times}
article, at
\url{http://www.nytimes.com/2009/08/06/technology/06stats.html},
profiling a top statistician at Google.  Also, watch the video at
\url{http://www.lecturemaker.com/tag/cran/} in which some engineers at
Google and Facebook explain how their firms use statistics and the R
language.  Google and Facebook are just examples.  There are many, many
other tech firms, big and small, that make extensive use of this
material.

\item Your ability to put the theory to good practical use will be
greatly enhanced by use of the R statistical programming language.  R
is the standard real-world statistical computational tool in use today.
(Some of you may have taken STA 32, which uses a bit of R; our usage
will go much further than that level.)

In fact, Google uses R extensively, and it has its own R coding style
guidelines
(\url{http://google-styleguide.googlecode.com/svn/trunk/google-r-style.html}).
They're not to my taste, but you can see that R is a big deal at Google.

There was a nice {\it New York Times} article on R; see
\url{http://www.nytimes.com/2009/01/07/technology/business-computing/07program.html}.

\item Probability and statistics play major roles in our daily lives, in
everything from Lake Tahoe casinos to buying insurance to voting in
elections.  Understanding these concepts enhances our lives.  ECS 132,
as a more practical, data-oriented course, better achieves this goal.

\end{itemize}

\section{Workload}

There will be approximately five assignments, consisting of mathematical
work plus some light programming work.  The math will be
intellectually similar in spirit to ``word problems'' in calculus.

{\bf IN ORDER TO ACHIEVE A DECENT GRADE, PLAN TO SPEND SIGNIFICANT TIME
ON CAREFUL READING OF THE TEXT, AT LEAST FIVE HOURS PER WEEK.}

All in all, the number of hours per week you'll put in should be similar
to something like ECS 60.  Note, though, that much of this will be group
work.

\input ../GenCourseInfo/Machines.tex
\input ../GenCourseInfo/LaTeX.tex
\input ../GenCourseInfo/SyllabusCore.tex

\section{Prerequisites}

The required background is:

\begin{itemize}

\item 
$\frac{d}{dt} \sin^2(t) = 2 \sin(t) \cos(t)$, 
\hspace{0.25in}
$\int_{0}^{\infty} \lambda e^{-\lambda t} ~ dt$ = 1,
\hspace{0.25in}
$\sum_{i=0}^{\infty} p^i = \frac{1}{1-p} ~~ (|p| < 1)$

Derivatives, integrals, infinite series.

\item 
$
\left (         
\begin{array}{cc}
a & b \\
c & d
\end{array}
\right )^{-1}     
\left (         
\begin{array}{c}
e \\
f
\end{array}
\right )
$

Basic matrix operations, i.e. addition, multiplication and inverse.

\item {\tt if (n > x+y) z = 168;}

Reasonable programming and debugging skill; basic awareness of the
concepts of bits/bytes, memory addresses and data structures; experience
in writing code to read and write files.

\end{itemize}


