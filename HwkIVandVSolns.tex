\documentclass{article}

\setlength{\oddsidemargin}{-0.5in}
\setlength{\evensidemargin}{-0.5in}
\setlength{\topmargin}{0.0in}
\setlength{\headheight}{0in}
\setlength{\headsep}{0in}
\setlength{\textwidth}{7.0in}
\setlength{\textheight}{9.5in}
\setlength{\parindent}{0in}
\setlength{\parskip}{0.05in}
\setlength{\columnseprule}{0.3pt}
\usepackage{fancyvrb}
\usepackage{hyperref}

\usepackage{listings,relsize}

\begin{document}

\lstset{language=R,basicstyle=\smaller}

{\bf 1.}  Let L denote the number of students who sit on the left, so
that the probability in question is P(L $>$ 0.6N).

\begin{itemize}

\item [(a)] The conditional distribution of L given N is binomial with
parameters N and 0.5, so use the binomial pmf.  Then follow the method
on p.50 for the approximate case.

\item [(b)] Break things down according to N:

\begin{eqnarray}
P(L > 0.6N) &=& \sum_{i=0}^{\infty} P(L > 0.6N \textrm{ and } N = i) \\ 
&=& \sum_{i=0}^{\infty} P(L > 0.6i) \cdot P(N = i) \\ 
\end{eqnarray}

Then plug in the binomial and Poisson pmfs.

\end{itemize}

{\bf 2.}  Take the matrix A to have (-1,1,0,0,0) in the first row,
(0,-1,1,0,0) in the second, etc.

{\bf 3.}  For a geometrically distributed random variable M with parameter
p, we have 

\begin{equation}
P(M > i) = P(\textrm{first i trials are failures}) = (1-p)^i 
\end{equation}

So, we hope to show that $q_i$ has that form.  Show first that $q_2$ has
that form, then show that $q_3$ does etc. to see the pattern.  Extra
Credit:  Formalize using math induction.

{\bf A.}  Note that this is ``event-oriented'' DES.  Here are the main
points:

\begin{itemize}

\item Set up event types for passenger arrival, bus arrival (specifying
the stop), and speed change.  

\item Each stop needs its own arrivals stream.  

\item Set up variables for:

   \begin{itemize}

   \item the current speed and position of the bus

   \item the number of passengers currently on the bus and their destinations

   \item the number of passengers currently waiting at each stop

   \end{itemize}

\end{itemize}

Dealing with events:

\begin{itemize}

\item Passenger arrival:  Increment the count of those waiting at that
stop.  Schedule the next arrival.

\item Bus arrival:  Any passengers whose destination is this stop will
alight.  Then as many as possible will board.  Update count of number
waiting, number on the bus, and the destinations of those on the bus.
Generate the next bus-arrival event (at the next stop).

\item Speed change:  Generate new speed.  Recompute the time of the
scheduled bus-arrival event for the next stop.  This most conveniently
done by removing that event from the event list and then reinserting it
with the new time.  Also schedule the time of the next speed change.

\end{itemize}

\end{document}

