\chapter{The Exponential Distributions}
\label{chap:expondistr}

The family of exponential distributions, Section \ref{exponfam}, has a
number of remarkable properties, which contribute to its widespread
usage in probabilistic modeling.  We'll discuss those here.

\section{Connection to the Poisson Distribution Family}
\label{connpoi}

Suppose the lifetimes of a set of light bulbs are independent and
identically distributed {\bf (i.i.d.)}, and consider the following
process.  At time 0, we install a light bulb, which burns an amount of
time $X_1$.  Then we install a second light bulb, with lifetime $X_2$.
Then a third, with lifetime $X_3$, and so on.  

Let 

\begin{equation}
\label{ti}
T_r = X_1+...+X_r
\end{equation}

denote the time of the $r^{th}$ replacement.  Also, let N(t) denote the
number of replacements up to and including time t.  Then it can be shown
that if the common distribution of the $X_i$ is exponentially
distributed, the N(t) has a Poisson distribution with mean $\lambda t$.
And the converse is true too:  If the $X_i$ are independent and
identically distributed and N(t) is Poisson, then the $X_i$ must have
exponential distributions.  In summary:

\begin{theorem}
Suppose $X_1, X_2, ...$ are i.i.d. nonnegative continuous random
variables.  Define

\begin{equation}
T_r = X_1+...+X_r
\end{equation}

and 

\begin{equation}
N(t) = \max\{k: T_k \leq t\}
\end{equation}

Then the distribution of N(t) is Poisson with parameter $\lambda t$ for
all t if and only if the $X_i$ have an exponential distribution with
parameter $\lambda$.

\end{theorem}

In other words, N(t) will have a Poisson distribution if and only if the
lifetimes are exponentially distributed.  

\begin{proof}

{\it ``Only if'' part:}

The key is to notice that the event $X_1 > t$ is exactly equivalent to
$N(t) = 0$.  If the first light bulb lasts longer than t, then the
count of burnouts at time t is 0, and vice versa.  Then

\begin{eqnarray}
\label{onlyif}
P(X_1 > t) &=& P[N(t) = 0] ~~ \textrm{(see above equiv.) } \\
&=& \frac{(\lambda t)^0}{0!} \cdot e^{-\lambda t} ~~ ((\ref{poispmf})\\
&=& e^{-\lambda t} 
\end{eqnarray}

Then

\begin{equation} 
f_{X_1}(t) = \frac{d}{dt} (1 - e^{-\lambda t}) =
\lambda e^{-\lambda t} 
\end{equation}

That shows that $X_1$ has an exponential distribution, and since the
$X_i$ are i.i.d., that implies that all of them have that distribution.

{\it ``If'' part:}

We need to show that if the $X_i$ are exponentially distributed with
parameter $\lambda$, then for u nonnegative and each positive integer k,

\begin{equation}
\label{poispmfcopy}
P[N(u) = k] = \frac{(\lambda u)^k e^{-\lambda u}}{k!}
\end{equation}

The proof for the case k = 0 just reverses (\ref{onlyif}) above.  The
general case, not shown here, notes that $N(u) \leq k$ is equivalent to
$T_{k+1} > u$.  The probability of the latter event can be found by
integrating (\ref{erldens}) from u to infinity.  One needs to perform
k-1 integrations by parts, and eventually one arrives at
(\ref{poispmfcopy}), summed from 1 to k, as required.

\end{proof}
  
The collection of random variables N(t) $t \geq 0$, is called a {\bf
Poisson process}.

% \begin{equation}
% \label{ntmean}
% E[N(t)] = \lambda t
% \end{equation}

The relation $E[N(t)] = \lambda t$ says that replacements are occurring
at an average rate of $\lambda$ per unit time.  Thus $\lambda$ is called
the {\bf intensity parameter} of the process.  It is this ``rate''
interpretation that makes $\lambda$ a natural indexing parameter in
(\ref{expdens}).

\section{Memoryless Property of Exponential Distributions}
\label{nomemxxx}

One of the reasons the exponential family of distributions is so famous
is that it has a property that makes many practical stochastic models
mathematically tractable:  The exponential distributions are
{\bf memoryless}.  

\subsection{Derivation and Intuition}

What the term {\it memoryless} means for a random variable W
is that for all positive t and u

\begin{equation}
\label{stmtnomemxxx}
P(W > t+u | W > t) = P(W > u)
\end{equation}

Any exponentially distributed random variable has this property.
Let's derive this:

\begin{eqnarray}
\label{nomemderivxxx}
P(W > t+u | W > t) &=& \frac{P(W > t+u {\rm ~and~} W > t)}{P(W > t)} \\ 
&=& \frac{P(W > t+u)}{P(W > t)} \\
&=&  \frac
{\int_{t+u}^{\infty} \lambda e^{-\lambda s}~ ds}
{\int_{t}^{\infty} \lambda e^{-\lambda s}~ ds} \\
&=& e^{-\lambda u} \\
&=& P(W > u)
\end{eqnarray}

We say that this means that ``time starts over'' at time t, or that W
``doesn't remember'' what happened before time t.

It is difficult for the beginning modeler to fully appreciate the
memoryless property.  Let's make it concrete.  Consider the problem of
waiting to cross the railroad tracks on Eighth Street in Davis, just
west of J Street. One cannot see down the tracks, so we don't know
whether the end of the train will come soon or not.

If we are driving, the issue at hand is whether to turn off the car's
engine.  If we leave it on, and the end of the train does not come for a
long time, we will be wasting gasoline; if we turn it off, and the end
does come soon, we will have to start the engine again, which also
wastes gasoline.  (Or, we may be deciding whether to stay there, or go
way over to the Covell Rd. railroad overpass.)

Suppose our policy is to turn off the engine if the end of the train
won't come for at least s seconds. Suppose also that we arrived at the
railroad crossing just when the train first arrived, and we have already
waited for r seconds.  Will the end of the train come within s more
seconds, so that we will keep the engine on? If the length of the train
were exponentially distributed (if there are typically many cars, we can
model it as continuous even though it is discrete), Equation
(\ref{stmtnomemxxx}) would say that the fact that we have waited r seconds so
far is of no value at all in predicting whether the train will end
within the next s seconds. The chance of it lasting at least s more
seconds right now is no more and no less than the chance it had of
lasting at least s seconds when it first arrived.

\subsection{Uniquely Memoryless}

By the way, the exponential distributions are the only continuous
distributions which are memoryless.  (Note the word {\it continuous}; in
the discrete realm, the family of geometric distributions are also
uniquely memoryless.) This too has implications for the theory.  A rough
proof of this uniqueness is as follows:

Suppose some continuous random variable V has the memoryless property,
and let R(t) denote $1-F_V(t)$.  Then from (\ref{stmtnomemxxx}), we would
have 

\begin{equation}
R(t+u)]/R(t) = R(u)
\end{equation} 

or

\begin{equation}
R(t+u) = R(t) R(u)
\end{equation}

Differentiating both sides with respect to t, we'd have

\begin{equation}
R\prime(t+u) = R\prime(t) R(u)  
\end{equation}

Setting t to 0, this would say

\begin{equation}
R\prime(u) = R\prime(0) R(u)
\end{equation}

or, setting $c = R(u),$

\begin{equation}
c = \frac{R'(t)}{R(t)} = \frac{d}{dt} [\ln(R(t)]
\end{equation}

Integrating both sides, we have

\begin{equation}
ct +b = \ln(R(t)
\end{equation}

for some constant $b$.  Exponentiating, we have

\begin{equation}
R(t) = e^{ct+b}
\end{equation}

We know that 

\begin{equation}
\lim_{t \rightarrow \infty} R(t) = 0
\end{equation}

That forces $c$ to be negative, and $b$ to be 0.  In other words,

\begin{equation}
R(t) = e^{-ht}  
\end{equation}

which is exactly 1 minus the cdf for an exponentially distributed random
variable.  

\subsection{Example:  ``Nonmemoryless'' Light Bulbs}

Suppose the lifetimes in years of light bulbs have the density 2t/15 on
(1,4), 0 elsewhere.  Say I've been using bulb A for 2.5 years now in a
certain lamp, and am continuing to use it.  But at this time I put a new
bulb, B, in a second lamp.  I am curious as to which bulb is more likely
to burn out within the next 1.2 years.  Let's find the two
probabilities.

For bulb A:

\begin{equation}
P(L > 3.7 | L > 2.5) =
\frac{P(L > 3.7)}{P(L > 2.5)} =
0.24
\end{equation}

For bulb B:

\begin{equation}
P(X > 1.2) = \int_{1.2}^{4} 2t/15 ~ dt = 0.97
\end{equation}

So you can see that the bulbs do have ``memory.''  We knew this
beforehand, since the exponential distributions are the only continuous
ones that have no memory.

\section{\label{min}Example:  Minima of Independent Exponentially
Distributed Random Variables}  
\label{minexp}

The memoryless property of the exponential distribution (Section
\ref{nomemxxx} leads to other key properties.  Here's a famous one:

\begin{theorem}
\label{thm1}

Suppose $W_{1},...,W_{k}$ are independent random variables, with $W_{i}$
being exponentially distributed with parameter $\lambda_{i}$.  Let
$Z=\min(W_{1},...,W_{k})$. Then Z too is exponentially distributed with
parameter $\lambda _{1}+...+\lambda_{k}$, and thus has mean equal to the
reciprocal of that parameter

\end{theorem}

{\bf Comments:}

\begin{itemize}

\item In ``notebook'' terms, we would have k+1 columns, one each for the
$W_i$ and one for Z.  For any given line, the value in the Z column will
be the smallest of the values in the columns for $W_1,...,W_k$; Z will
be equal to one of them, but not the same one in every line.  Then for
instance $P(Z = W_3)$ is interpretable in notebook form as the long-run
proportion of lines in which the Z column equals the $W_3$ column.

\item It's pretty remarkable that the minimum of independent exponential
random variables turns out again to be exponential.  Contrast that with
Section \ref{minunif}, where it is found that the minimum of independent
uniform random variables does NOT turn out to have a uniform
distribution.

\item The sum $\lambda_1+...+\lambda_n$ in (a) should make good
intuitive sense to you, for the following reasons.  Recall from Section
\ref{connpoi} that the parameter $\lambda$ in an exponential
distribution is interpretable as a ``light bulb burnout rate.''

Say we have persons 1 and 2.  Each has a lamp.  Person i uses Brand i
light bulbs, i = 1,2.  Say Brand i light bulbs have exponential
lifetimes with parameter $\lambda_i$.  Suppose each time person i
replaces a bulb, he shouts out, ``New bulb!'' and each time {\it anyone}
replaces a bulb, I shout out ``New bulb!'' Persons 1 and 2 are shouting
at a rate of $\lambda_1$ and $\lambda_2$, respectively, so I am shouting
at a rate of $\lambda_1 + \lambda_2$.

\end{itemize}

\begin{proof}

\begin{eqnarray}
F_{Z}(t) &=& P(Z \leq t) ~~~~ \textrm{(def. of cdf) }\\
&=& 1-P(Z>t) \\
&=& 1-P(W_{1}>t\textrm{ and }...\textrm{ and }W_{k}>t)  ~~~~
\textrm{(min $> t$ iff all $W_i > t$)} \\
&=& 1-\Pi_{i}\textrm{ } P(W_i > t)  ~~~~ \textrm{(indep.) }\\
&=& 1-\Pi_{i}\textrm{ }e^{-\lambda_{i}t} ~~~~ \textrm{ (expon. distr.)} \\
&=& 1 - e^{-(\lambda_1+...+\lambda_n) t}
\end{eqnarray}

Taking $\frac{d}{dt}$ of both sides proves the theorem.

\end{proof}

Also:

\begin{theorem}
Under the conditions in Theorem \ref{thm1}, 

\begin{equation}
\label{whichmin}
P(W_i < W_1,...,W_{i-1}, W_{i+1},...,W_k) =
\frac{\lambda_i}{\lambda_1+...+\lambda_k}
\end{equation}

(There are k terms in the denominator, not k-1.)

\end{theorem}

Equation (\ref{whichmin}) should be intuitively clear as well from the
above ``thought experiment'' (in which we shouted out ``New bulb!''):
On average, we have one new Brand 1 bulb every $1/\lambda_1$ time, so in
a long time t, we'll have about $t \lambda_1$ shouts for this brand.
We'll also have about $t \lambda_2$ shouts for Brand 2.  So, a
proportion of about

\begin{equation}
\frac{t\lambda_1}{t\lambda_1 + t\lambda_2}
\end{equation}

of the shots are for Brand 1.  Also, at any given time, the memoryless
property of exponential distributions implies that the time at which I
shout next will be the {\it minimum} of the times at which persons 1 and
2 shout next.  This intuitively implies (\ref{whichmin}).

\begin{proof}

Again consider the case k = 2, and then use induction.

Let $Z = \min(W_1,W_2)$ as before.  Then, with $s$ and $t$ playing the
roles of $W_1$ and $W_2$, we have

\begin{equation}
P(Z = W_1) = P(W_1 < W_2) -
\int_{0}^{\infty} \int_{s}^{\infty} 
\lambda_1 e^{-\lambda_1 s}   
\lambda_2 e^{-\lambda_2 t}   
~ dt ~ ds
\end{equation}

So,

\begin{equation}
P(Z = W_1) = \frac{\lambda_1}{\lambda_1 + \lambda_2}
\end{equation}

as claimed.

\end{proof}

% {\bf Note carefully:}  Just as the probability that a continuous random
% variable takes on a specific value is 0, the probability that two
% continuous and independent random variables are equal to each other is
% 0.  Thus in the above analysis, $P(W_1 = W_2) = 0$.

This property of minima of independent exponentially-distributed random
variables developed in this section is key to the structure of
continuous-time Markov chains, in Chapter \ref{conmarkov}.

\subsection{Example:  Computer Worm}
\label{senthi}

A computer science graduate student at UCD, Senthilkumar Cheetancheri,
was working on a worm alert mechanism.  A simplified version of the
model is that network hosts are divided into groups of size g, say on
the basis of sharing the same router.  Each infected host tries to
infect all the others in the group.  When g-1 group members are
infected, an alert is sent to the outside world.

The student was studying this model via simulation, and found some
surprising behavior.  No matter how large he made g, the mean time until
an external alert was raised seemed bounded.  He asked me for advice.

I modeled the nodes as operating independently, and assumed that if node
A is trying to infect node B, it takes an exponentially-distributed
amount of time to do so.  This is a continuous-time Markov chain.
Again, this topic is much more fully developed in Chapter
\ref{conmarkov}, but all we need here is the result of Section
\ref{minexp}, that exponential distributions are ``memoryless.''. 

In state i, there are i infected hosts, each trying to infect all of the
g-i noninfected hosts.  When the process reaches state g-1, the process
ends; we call this state an {\bf absorbing state}, i.e. one from which
the process never leaves.

Scale time so that for hosts A and B above, the mean time to infection
is 1.0.  Since in state i there are i(g-i) such pairs, the time to the
next state transition is the minimum of i(g-i) exponentially-distributed
random variables with mean 1.  Theorem \ref{thm1} tells us that this
minimum is also exponentially distributed, with parameter $i(g-i) \cdot
1$.  Thus the mean time to go from state i to state i+1 is 1/[i(g-i)].

% \checkpoint

Then the mean time to go from state 1 to state g-1 is

\begin{equation}
\label{s1}
\sum_{i=1}^{g-1} \frac{1}{i(g-i)}
\end{equation}

Using a calculus approximation, we have

\begin{equation}
\int_1^{g-1} \frac{1}{x(g-x)} ~ dx = \frac{1}{g} \int_1^{g-1}
(\frac{1}{x} +
\frac{1}{g-x}) ~ dx = \frac{2}{g} \ln(g-1)
\end{equation}

The latter quantity goes to zero as $g \rightarrow \infty$.  This
confirms that the behavior seen by the student in simulations holds in
general.  In other words, (\ref{s1}) remains bounded as $g \rightarrow
\infty$.  This is a very interesting result, since it says that the mean
time to alert is bounded no matter how big our group size is.

So, even though our model here was quite simple, probably overly so, it
did explain why the student was seeing the surprising behavior in his
simulations.

\subsection{Example:  Electronic Components}

Suppose we have three electronic parts, with independent lifetimes that
are exponentially distributed with mean 2.5.  They are installed
simultaneously.  Let's find the mean time until the last failure occurs.

Actually, we can use the same reasoning as for the computer worm example
in Section \ref{senthi}:  The mean time is simply

\begin{equation}
1/(3 \cdot 0.4) + 1/(2 \cdot 0.4) + 1/(1 \cdot 0.4)
\end{equation}

\section{A Cautionary Tale:  the Bus Paradox}
\label{busparadox}

Suppose you arrive at a bus stop, at which buses arrive according to a
Poisson process with intensity parameter 0.1, i.e. 0.1 arrival per
minute.  Recall that the means that the interarrival times have an
exponential distribution with mean 10 minutes. What is the expected
value of your waiting time until the next bus?

Well, our first thought might be that since the exponential distribution
is memoryless, ``time starts over'' when we reach the bus stop.
Therefore our mean wait should be 10.

On the other hand, we might think that on average we will arrive halfway
between two consecutive buses. Since the mean time between buses is 10
minutes, the halfway point is at 5 minutes. Thus it would seem that our
mean wait should be 5 minutes.

Which analysis is correct? Actually, the correct answer is 10 minutes.
So, what is wrong with the second analysis, which concluded that the
mean wait is 5 minutes?  The problem is that the second analysis did not
take into account the fact that although inter-bus intervals have an
exponential distribution with mean 10, \textit{the particular inter-bus
interval that we encounter is special.}

\subsection{Length-Biased Sampling}
\label{lbsamp}

Imagine a bag full of sticks, of different lengths. We reach into the
bag and choose a stick at random. The key point is that not all pieces
are equally likely to be chosen; the longer pieces will have a greater
chance of being selected.

Say for example there are 50 sticks in the bag, with ID numbers from 1
to 50.  Let X denote the length of the stick we obtain if select a stick
on an equal-probability basis, i.e. each stick having probability 1/50
of being chosen.  (We select a random number I from 1 to 50, and choose
the stick with ID number I.)  On the other hand, let Y denote the length
of the stick we choose by reaching into the bag and pulling out
whichever stick we happen to touch first.  Intuitively, the distribution
of Y should favor the longer sticks, so that for instance $EY > EX$.

Let's look at this from a ``notebook'' point of view.  We pull a stick
out of the bag by random ID number, and record its length in the X
column of the first line of the notebook.  Then we replace the stick,
and choose a stick out by the ``first touch'' method, and record its
length in the Y column of the first line.  Then we do all this again,
recording on the second line, and so on.  Again, because the ``first
touch'' method will favor the longer sticks, the long-run average of the
Y column will be larger than the one for the X column.


Another example was suggested to me by UCD grad student Shubhabrata
Sengupta.  Think of a large parking lot on which hundreds of buckets are
placed of various diameters.  We throw a ball high into the sky, and see
what size bucket it lands in.  Here the density would be proportional to
area of the bucket, i.e. to the square of the diameter. 

Similarly, the particular inter-bus interval that we hit is likely to be
a longer interval.  To see this, suppose we observe the comings and
goings of buses for a very long time, and plot their arrivals on a time
line on a wall.  In some cases two successive marks on the time line are
close together, sometimes far apart.  If we were to stand far from the
wall and throw a dart at it, we would hit the interval between some pair of
consecutive marks.  Intuitively we are more apt to hit a wider interval
than a narrower one.

The formal name for this is \textbf{length-biased sampling}.

Once one recognizes this and carefully derives the density of that
interval (see below), we discover that that interval does indeed tend to
be longer---so much so that the expected value of this interval is 20
minutes!  Thus the halfway point comes at 10 minutes, consistent with
the analysis which appealed to the memoryless property, thus resolving
the ``paradox.''

In other words, if we throw a dart at the wall, say, 1000
times, the mean of the 1000 intervals we would hit would be about 20.
This in contrast to the mean of all of the intervals on the wall, which
would be 10.

\subsection{Probability Mass Functions and Densities in Length-Biased
Sampling}

Actually, we can intuitively reason out what the density is of the
length of the particular inter-bus interval that we hit, as follows.

First consider the bag-of-sticks example, and suppose (somewhat
artificially) that stick length X is a discrete random variable.  Let Y
denote the length of the stick that we pick by randomly touching a stick
in the bag.  

Again, note carefully that for the reasons we've been discussing here,
the distributions of X and Y are different.  Say we have a list of all
sticks, and we choose a stick at random from the list.  Then the length
of that stick will be X.  But if we choose by touching a stick in the
bag, that length will be Y.

Now suppose that, say, stick lengths 2 and 6 each comprise 10\% of the
sticks in the bag, i.e.

\begin{equation}
p_X(2) = p_X(6) = 0.1
\end{equation}

Intuitively, one would then reason that 

\begin{equation}
p_Y(6) = 3 p_Y(2) 
\end{equation}

In other words, even though the sticks of length 2 are just as numerous
as those of length 6, the latter are three times as long, so they should
have triple the chance of being chosen.  So, the chance of our choosing
a stick of length j depends not only on $p_X(j)$ but also on j itself.

We could write that formally as

\begin{equation}
p_Y(j) \propto j p_X(j)
\end{equation}

where $\propto$ is the ``is proportional to'' symbol.  Thus

\begin{equation}
p_Y(j) = c j p_X(j)
\end{equation}

for some constant of proportionality c.

But a probability mass function must sum to 1.  So, summing over all
possible values of j (whatever they are), we have

\begin{equation}
1 = \sum_{j} p_Y(j) = \sum_{j} c j p_X(j) 
\end{equation}

That last term is c E(X)!  So, c = 1/EX, and

\begin{equation}
\label{lbs}
p_Y(j) = \frac{1}{EX} \cdot j p_X(j)
\end{equation}

% Note that this is not some absolute physical law.  Different people
% might draw sticks from the bag in different ways.  But it is a
% reasonable model.

The continuous analog of (\ref{lbs}) is

\begin{equation}
\label{lbscontin}
f_Y(t) = \frac{1}{EX} \cdot t f_X(t)
\end{equation}

So, for our bus example, in which $f_X(t) = 0.1e^{-0.1 t}, ~ t > 0$ and
EX = 10,

\begin{equation}
f_Y(t) = 0.01 t e^{-0.1 t}
\end{equation}

You may recognize this as an Erlang density with r = 2 and $\lambda =
0.1$.  That distribution does indeed have mean 20, consistent with the
discussion at the end of Section \ref{lbsamp}.
