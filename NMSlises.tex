\documentclass{beamer}  
\usepackage{graphicx}
\usepackage{url}
\usepackage{fancyvrb}
\usepackage[pdftex]{color}

\mode<presentation>
{ \usetheme{Darmstadt} }

\title{Rdsm: Distributed (Quasi-)Threads Programming in R}

\author{
Norm Matloff 
Department of Computer Science \\
University of California at Davis \\
Davis, CA 95616 USA \\
matloff@cs.ucdavis.edu 
}

\date{}
 
\begin{document} 

\begin{frame}
\titlepage
\end{frame}

\begin{frame}
\frametitle{Parallel R}


\begin{columns}

   \begin{column}{0.3\textwidth}
   \includegraphics[width=1.25in]{RailwayTracks.jpg}
   \end{column}

   \begin{column}{0.7\textwidth}

   \begin{itemize}

   \item Many excellent packages are available.
   \pause

   \item But most use message-passing paradigm or variants, e.g. Rmpi, snow.
   \pause

   \item True shared-memory choices very limited.
   \pause

      \begin{itemize}
      \item bigmemory
      \item attached C (OpenMP, CUDA)
      \end{itemize}

   \end{itemize}

   \end{column}

\end{columns}

\end{frame}

\begin{frame}
\frametitle{!`{\it Arriba} sharing!}
\pause

\begin{columns}

   \begin{column}{0.3\textwidth}
   \includegraphics[width=1.5in]{Sharing.jpg}
   \end{column}

   \begin{column}{0.7\textwidth}
   \begin{itemize}
   \item Many in the parallel processing community consider shared-memory 
   paradigm to be clearer, more concise, e.g. Chandra (2001), Hess 
   (2002).
   \pause
   \item Conversion from sequential code easier than in message-passing
   case.
   % \pause
   % \item Familiar threads model.
   \end{itemize}
   \end{column}

\end{columns}

\end{frame}

\begin{frame}
\frametitle{Why Threads?}

\includegraphics[width=1.25in]{Spools.jpg}

My definition here: Concurrent processes, communicating through shared
memory.
\pause

\begin{itemize}

\item Enable parallel computation.
\pause

   \begin{itemize}

   \item Standard approach for speedup on shared-memory machines.
   \pause

   \end{itemize}

\item Enable parallel {\bf I/O}!
\pause

   \begin{itemize}
    
   \item Perhaps less well-known, more commonly used.
   \pause

   \item E.g. Web servers.
   \end{itemize}

\end{itemize}

\end{frame}

\begin{frame}
\frametitle{Rdsm:  History and Motivation}
\pause

\begin{itemize}

\item Goals:
\pause

   \begin{itemize}

   \item Shared-memory vehicle for R, focused on parallel programming.
   \pause

   \item Distributed computing capability, e.g. for collaborative tools.
   \pause

   \end{itemize}

\item Easy to build on my previous product, PerlDSM (Matloff, 2002).

\end{itemize}

\end{frame}

\begin{frame}
\frametitle{What Is Rdsm?}
\pause

\begin{itemize}

\item Provides R programmers with a threads-like programming
environment:
\pause

   \begin{itemize}

   \item Multiple R processes.
   \pause
   
   \item Read/write shared variables, accessed through ordinary R syntax.
   \pause

   \item Locks, barriers, wait/signal, etc.

   \end{itemize}
   \pause

\item  Platforms:

   \begin{columns}
      \begin{column}{0.3\textwidth}
      \includegraphics[width=1.5in]{Glob.jpg}
      \end{column}
      \begin{column}{0.5\textwidth}
      Processes can be on the same mulicore machine \underline{or} on 
      distributed, geographically disperse machines.
      \end{column}
   \end{columns}

\end{itemize}

\end{frame}

\begin{frame}
\frametitle{Applications of Rdsm}
\pause

\begin{itemize}

\item Performance programming, in ``embarassingly parallel''
settings.  
\pause

(Possibly the limit for any parallel R.)
\pause

\item Parallel I/O applications, e.g. parallel collection of Web data and
its concurrent statistical analysis.
\pause

\item Collaborative tools.
\pause

\item Even games!

\end{itemize}
\end{frame}

\begin{frame}[fragile]
\frametitle{What Does Rdsm Code Look Like?}
\pause

Answer:  Except for initialization, it looks just like---and
IS---ordinary R code.
\pause

For example, to replace the 3rd column of a shared matrix {\bf m} by a
vector of all 1s:

\begin{verbatim}
m[,3] <- 1  # use recycling
\end{verbatim}
\pause

This is ordinary, garden-variety R code.
\pause

And it IS shared: If process 3 executes the above and then process 8
does

\begin{verbatim}
x <- m[2,3]
\end{verbatim}

then {\bf x} will be 1 at process 8.

\end{frame}

\begin{frame}[fragile]
\frametitle{What Does Rdsm Code Look Like? (cont'd.)}

The only difference is in creating the variable:
\pause

\begin{verbatim}
# create shared 6x6 matrix
newdsm("m","dsmm","double",size=c(6,6))  
\end{verbatim}

Note the special {\bf "dsmm"} class for shared matrices.
\pause
(Also have classes for shared vectors and lists.)
\pause

Otherwise, it's ordinary R syntax, with threads.

\end{frame}

\end{document}

