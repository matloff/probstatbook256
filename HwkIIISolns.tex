\documentclass{article}

\setlength{\oddsidemargin}{-0.5in}
\setlength{\evensidemargin}{-0.5in}
\setlength{\topmargin}{0.0in}
\setlength{\headheight}{0in}
\setlength{\headsep}{0in}
\setlength{\textwidth}{7.0in}
\setlength{\textheight}{9.5in}
\setlength{\parindent}{0in}
\setlength{\parskip}{0.05in}
\setlength{\columnseprule}{0.3pt}
\usepackage{fancyvrb}
\usepackage{hyperref}

\usepackage{listings,relsize}

\begin{document}

\lstset{language=R,basicstyle=\smaller}

Name: \_\_\_\_\_\_\_\_\_\_\_\_\_\_\_\_\_\_\_\_\_\_\_\_\_\_\_\_

Directions: {\bf Work only on this sheet} (on both sides, if needed); do not
turn in any supplementary sheets of paper. There is actually plenty of room
for your answers, as long as you organize yourself BEFORE starting writing.

{\bf \Large SHOW YOUR WORK!}  

{\bf 1.} 

\begin{itemize}

\item [(a)] States are 0..3, so the matrix is 3x3.  For example,

\begin{equation}
p_{21} = P(\textrm{1 leaves, 0 board or 2 leave, 1 board})
= 2(0.2)(0.8)(0.5) + (0.2)^2 (0.4)
\end{equation}

\item [(b)] The function code will first form the transition matrix, and
then call {\bf findpi1()}, as on p.58.

\item [(c)] 

\begin{equation}
\sum_{i=0}^c i \pi_i
\end{equation}

\item [(e)] Per the hint, use the approach in (10.43).  One bus stop is
analogous to one time slot in ALOHA.  We need to find

\begin{equation}
\frac
{E(\textrm{number of passengers who can't board})}
{E(\textrm{number of passengers at the stop})}
\end{equation}

The numerator is 

\begin{equation}
\pi_{c} (1 \cdot 0.4 + 2 \cdot 0.1)  +
\pi_{c-1} (1 \cdot 0.1)  
\end{equation}

The denominator is (0.5) 0 + (0.4) 1 + (0.1) 2.

\end{itemize}

{\bf 2.}

\begin{itemize}

\item [(a)] We have an Erlang distribution with r = 5 and $\lambda = 1/100$.
Use (4.59).

\item [(b)] Erlang is a special case of the gamma distribution, so use
{\bf pgamma()} with the above parameter values.

\end{itemize}

{\bf 3.}

\begin{itemize}

\item [(a)]

\begin{Verbatim}[fontsize=\relsize{-2},numbers=left]
# inputs links, a 2-column matrix or data frame, and returns the degree
# distribution, the i-th element of which is the count of the number of
# nodes with i links; graph is assume to be directed; node ID numbers
# are assume to be consecutive integers starting from 1
dd <- function(links) {
   nn <- max(links)  # number of nodes
   # set up adjacency matrix
   adj <- matrix(rep(0,nn^2),nrow=nn)
   for (i in 1:nrow(links)) {
      from <- links[i,1]
      to <- links[i,2]
      adj[from,to] <- 1
   }
   degrees <- apply(adj,1,sum)
   # use histogram drawing ftn to get counts
   maxd <- max(degrees)
   brks <- seq(-0.1,maxd+0.9,1)  # define the histo bins
   h <- hist(degrees,breaks=brks,plot=F)
   return(h$counts)
}
\end{Verbatim}

\item [(b)]

Straightforward, following the instructions in the problem specs.  Get
the slope from element 2 of the {\bf coef} component of the output of
{\bf lm()}.

\item [(c)] The power law seems to be very poor model.

\end{itemize}

{\bf 4.}  The exponential model seems to fit well here.

{\bf 5.}  Let $U_1,...,U_n$ be the errors, and let T be their total.
They are assumed i.i.d.  U(-0.5,0.5).

\begin{itemize}

\item [(a)] Due to the uniformity, the probability is 2(0.5-0.3) = 0.4.

\item [(b)] Use the Central Limit Theorem.  T is approximately normally
distributed with mean ET = 0 and variance equal to n times the variance
of U(-0.5,0.5), which is 1/12 from the book.

\item [(c)] This is (b) in reverse.  We want to find c such that

\begin{equation}
0.95 \approx P(-c < T < c)
\end{equation}

Again use the CLT.

\end{itemize}

\end{document}

